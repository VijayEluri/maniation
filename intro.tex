This project is concerned with the dynamics, i.e.\ the behaviour over time, of a general class
of mechanical systems. In this report I describe the development and evaluation of a program to
simulate such physical systems. The primary domain of this work is computer graphics, where
physically-based modelling is a useful tool for creating realistic animations, although a wider
range of applications in robotics and engineering can be envisaged.

\section{Motivation}

3D animated graphics have become an everyday part of our lives through films, advertising and
computer games. Recent years have seen not only the production of feature-length animated films,
but also the widespread use of animation in combination with traditionally shot footage. The
techniques are now so refined that computer generated and recorded pictures are sometimes
indistinguishable even to experts, thus opening up a whole range of new artistic possibilities.

Currently computer animation still involves a large amount of manual effort, despite commonly
being termed ``computer generated''. It is said that in large-budget film productions every single
frame is edited by hand, and many animations are completely hand-crafted. There are artistic
reasons for this~-- sometimes an animation is deliberately required to be physically impossible
to achieve a special effect~-- and also economic reasons (animators are cheaper to hire than
computer scientists). However, as the complexity of animated scenes increases, manual animation
becomes unfeasible. Large crowds, fluids, cloths and some types of deformable bodies are therefore
commonly animated using simulation techniques.

Unfortunately such scenes with an unbounded number degrees of freedom are difficult to simulate
well. Fluid dynamics is very complicated and a subject of ongoing research; the physics of
deformable bodies (so-called ``soft matter'') is not even completely understood. Graphical
simulations in these areas tend to resort to means which look good but are physically meaningless.
This project aims to address the more well-defined subject of character animation, i.e.\ the
movement of humans, many species of animal, and some types of alien creatures. Their common
feature is that their bodies are deformable on the basis of a skeleton. Such bodies are called
\emph{articulated bodies} in computer graphics, and it is usually assumed that this is the only
type of deformation~-- they do not have ``soft flesh''. \textsl{Skeleton} is not used as an
anatomically accurate term, because more or fewer bones might be used, depending on the animation
requirements.

Rigid body dynamics, which will form the basis for physically based character animation, is well
understood and has a solid theoretical underpinning against which the simulation results can be
checked. Hence physically-based modelling lends itself more to an objective evaluation than
most other areas of computer graphics, making it well suited as a Part~II project.

\section{Scope and requirements}

Simulations of dynamic systems are employed in a wide range of environments:
\begin{enumerate}
\item games and related real-time applications, where fast computation is paramount and anything
    which ``looks good'' is acceptable;
\item professional animation systems for high-quality graphics, in which calculations are done
    offline;
\item engineering and quantitative research, where accuracy and knowledge of errors are the main
    concerns.
\end{enumerate}

The project proposal identifies itself with the second area, a trade-off between accuracy and
speed. Once the scope was set, a more detailed analysis of the requirements had to be undertaken.
Based on the project proposal, the program must
\begin{itemize}
\item read several polygon meshes, with skeletons and physical properties, from a file;
\item provide a general framework for handling interactions between objects;
\item ensure, as a particular type of interaction, that rigid bodies do not penetrate each other;
\item combine forces, torques and impulses from interactions and hence calculate the change to
    the system over time;
\item output the simulation state over time in rendered form on screen, and into a file such that
    it can be processed by other applications.
\end{itemize}

With respect to physical laws, the simulation should include
\begin{itemize}
\item the momentary state of a rigid body (position and orientation) and its velocities (linear
    and angular),
\item changes to a body's state through forces, torques, linear and angular impulses,
\item inertial mass, gravity and collisions between bodies,
\item appropriate handling for articulated bodies, including limits on the range of valid angles
    for each joint. (This point is deliberately vague because the different possible approaches
    are discussed later in this report.)
\end{itemize}

This list of requirements differs slightly from those stated in the proposal:
\begin{itemize}
\item The simulation of friction forces was removed from the requirements, because the literature
    suggested that it was a very difficult problem and I decided it was beyond the scope of a
    Part~II project.
\item Muscular forces are easy to implement but involve a large amount of manual effort to
    control, so I shifted the emphasis towards ``passive'' systems with no external forces apart
    from gravity.
\item The requirements were extended from one body to multiple bodies.
\item Output of the results to a file was added to ease evaluation.
\end{itemize}

\section{Overview}

The dynamics of rigid bodies are physically described by ordinary differential equations (ODEs),
and a simulation essentially comes down to numerically finding their solution, given a set of
initial conditions. Hence this project can be broken into five components:

\begin{enumerate}
\item A numerical solver for ordinary differential equations. This is a well-researched area, and
    a range of good standard algorithms exist.
\item An implementation of the equations of motion for rigid bodies. This is standard physics but
    has some implementation quirks.
\item Algorithms to ensure the correct handling of articulated bodies. This is by far the most
    challenging part of the project.
\item Detection of collision between bodies. A vast amount of research has been done in this area,
    and most algorithms are concerned with speed. However, since run-time efficiency is not my
    primary concern, this part was kept quite simple.
\item Handling of collisions, that is computation of the correct impulses between colliding
    bodies. This must also work in the context of articulated bodies!
\end{enumerate}

Each of these components will be discussed in detail in the following chapters.
