\section{Quaternion integration\label{quatProofs}}
\subsection{Conservation of magnitude\label{quatIntegrationMagnitude}}
Define the quaternion dot product, in analogy to the 4D vector dot product, to be
\begin{equation}
\q{p}\cdot\q{q} =
    (p_w + p_x\qi + p_y\qj + p_z\qk)\cdot (q_w + q_x\qi + q_y\qj + q_z\qk) =
    p_w q_w + p_x q_x + p_y q_y + p_z q_z
\end{equation}
The dot product is commutative, contrary to the `normal' quaternion product.

The instantaneous rate of change is given to be
\begin{eqnarray*}
\dot{\q{q}} & = & \frac{1}{2}\ve{\omega}\q{q} =
    \frac{1}{2}(\omega_1\qi + \omega_2\qj + \omega_3\qk)
    (q_w + q_x\qi + q_y\qj + q_z\qk) \\*
& = & \frac{1}{2} ( - \omega_1 q_x - \omega_2 q_y - \omega_3 q_z ) +
    \frac{\qi}{2} ( \omega_1 q_w + \omega_2 q_z - \omega_3 q_y ) + \\*
&&  \frac{\qj}{2} (-\omega_1 q_z + \omega_2 q_w + \omega_3 q_x ) +
    \frac{\qk}{2} ( \omega_1 q_y - \omega_2 q_x + \omega_3 q_w )
\end{eqnarray*}

We now treat \q{q} and $\dot{\q{q}}$ as 4D vectors and calculate
their dot product:
\begin{eqnarray*}
\q{q}\cdot\dot{\q{q}} & = & \frac{1}{2} (
    - q_w \omega_1 q_x - q_w \omega_2 q_y - q_w \omega_3 q_z
    + q_x \omega_1 q_w + q_x \omega_2 q_z - q_x \omega_3 q_y \\*
&&  - q_y \omega_1 q_z + q_y \omega_2 q_w + q_y \omega_3 q_x
    + q_z \omega_1 q_y - q_z \omega_2 q_x + q_z \omega_3 q_w ) \\*
& = & 0.
\end{eqnarray*}
The rate of change is orthogonal to \q{q}, and therefore it is always
a tangent to the sphere, touching it at the point corresponding to \q{q}.

We can determine the magnitude $\norm{\dot{\q{q}}}$ from the sum of squares of the
components given above and find it to be
$\norm{\dot{\q{q}}} = \frac{1}{2}\norm{\ve{\omega}}\,\norm{\q{q}}$. Since we always
require $\q{q}$ to be a unit quaternion, we can reduce this to
\begin{equation}
\label{quatRateOfChangeMagnitude}
\norm{\dot{\q{q}}} = \frac{1}{2}\norm{\ve{\omega}}.
\end{equation}

Now let us determine what happens if we calculate $\q{q} + h\dot{\q{q}}$ for some finite $h$.
Note that this operation is required by all common numerical solvers of differential equations.
Consider the magnitude of the result:
\begin{eqnarray*}
\norm{\q{q} + h\dot{\q{q}}}^2 & = & (\q{q} + h\dot{\q{q}})\cdot(\q{q} + h\dot{\q{q}}) \\
&=& \q{q}\cdot\q{q} + 2h\q{q}\cdot\dot{\q{q}} + h^2\dot{\q{q}}\cdot\dot{\q{q}} \\
&=& 1 + 0 + \frac{h^2}{4}\norm{\ve{\omega}}^2 \\
&>& 1 \quad\quad\mbox{whenever}\quad \norm{\ve{\omega}} > 0.
\end{eqnarray*}

Hence, if the body in question is rotating, it is not possible for a standard numerical ODE solver
to preserve a quaternion's property of unit magnitude.

\subsection{Corrected quaternion integration\label{quatIntegrationDerivation}}

Assume that the body we are simulating is rotating at a constant angular velocity.
(This assumption is later weakened by the use of a more sophisticated ODE solver,
but for now we will stick with Euler's method.) Furthermore assume without loss of
generality that the body is rotating clockwise about its $x$ axis, which corresponds
to the world's $x$ axis. Then the orientation of the body (the quaternion describing
the linear transformation to get from the body's frame of reference into the world
frame) is given as a function of time by
\begin{equation}
\label{quatDerivationExact}
\q{q}(t) = \cos\left(\frac{\norm{\ve{\omega}}t}{2}\right) +
    \sin\left(\frac{\norm{\ve{\omega}}t}{2}\right)\qi
\end{equation}
(cf.\ equation~\ref{quatRotation}) and its angular velocity is
\begin{equation}
\ve{\omega} = (\omega_1, \omega_2, \omega_3)^T = (\norm{\ve{\omega}}, 0, 0)^T
\end{equation}
for some arbitrary $\norm{\ve{\omega}}$, measured in radians per unit time.

Now assume w.l.o.g. that we take a time step from $t = 0$ to $t = \Delta t$.
Then we require that the result returned by Euler's method for $\q{q}(\Delta t)$
after re-normalization be equal to its exact value in equation~\ref{quatDerivationExact}:
\begin{equation}
\label{quatDerivationSetup}
\cos\left(\frac{\norm{\ve{\omega}}\Delta t}{2}\right) +
    \sin\left(\frac{\norm{\ve{\omega}}\Delta t}{2}\right)\qi =
    \frac{\q{q}(0) + \Delta t \dot{\q{q}}(0)}
        {\norm{\q{q}(0) + \Delta t \dot{\q{q}}(0)}}
\end{equation}

We know from examining the 4D geometry that the value assigned to $\dot{\q{q}}$
in equation~\ref{quatRateOfChange} has the correct direction and merely needs to be
corrected in magnitude. In other words, we are searching for a scalar function
$f(\Delta t, \norm{\ve{\omega}})$ which will allow $\dot{\q{q}}$ to satisfy
equation~\ref{quatDerivationSetup}:
\begin{equation}
\dot{\q{q}}_{\Delta t}(t) = f\tilde{\ve{\omega}}(t)\q{q}(t)
\end{equation}

Observe that under the above assumptions $\q{q}(0) = 1$, and thus
$\dot{\q{q}}_{\Delta t}(0) = f \norm{\ve{\omega}} \qi$. Substituting this
into equation~\ref{quatDerivationSetup} and considering only the real part:
\begin{eqnarray*}
&& \cos\left(\frac{\norm{\ve{\omega}}\Delta t}{2}\right) =
    \left[1 + \left( f \norm{\ve{\omega}}\Delta t \right)^2 \right]^{-\frac{1}{2}} \\
&\Leftrightarrow&
    \left( f \norm{\ve{\omega}}\Delta t \right)^2 =
    \frac{1}{\cos^2\left(\frac{\norm{\ve{\omega}}\Delta t}{2}\right)} - 1 \\
&\Leftrightarrow&
    f(\Delta t, \norm{\ve{\omega}}) =
    \frac{1}{\norm{\ve{\omega}}\Delta t} \sqrt{\frac{
        1 - \cos^2\left(\frac{\norm{\ve{\omega}}\Delta t}{2}\right)}{
        \cos^2\left(\frac{\norm{\ve{\omega}}\Delta t}{2}\right)}} =
    \frac{1}{\norm{\ve{\omega}}\Delta t}
        \tan\left(\frac{\norm{\ve{\omega}}\Delta t}{2}\right)
\end{eqnarray*}

To check, we substitute this result into the $\qi$-imaginary part of
equation~\ref{quatDerivationSetup}:
\begin{eqnarray*}
\sin\left(\frac{\norm{\ve{\omega}}\Delta t}{2}\right) & = &
    \tan\left(\frac{\norm{\ve{\omega}}\Delta t}{2}\right)
    \left[ 1 + \tan^2\left(\frac{\norm{\ve{\omega}}\Delta t}{2}\right)
    \right]^{-\frac{1}{2}} \\
&=& \tan\left(\frac{\norm{\ve{\omega}}\Delta t}{2}\right)
    \left[ \frac{\cos^2\left(\frac{\norm{\ve{\omega}}\Delta t}{2}\right) +
    \sin^2\left(\frac{\norm{\ve{\omega}}\Delta t}{2}\right) }{
    \cos^2\left(\frac{\norm{\ve{\omega}}\Delta t}{2}\right) }
    \right]^{-\frac{1}{2}} \\
&=& \tan\left(\frac{\norm{\ve{\omega}}\Delta t}{2}\right)
    \cos\left(\frac{\norm{\ve{\omega}}\Delta t}{2}\right) \\
&=& \sin\left(\frac{\norm{\ve{\omega}}\Delta t}{2}\right)
\end{eqnarray*}

Thus we establish the validity of this expression for $f$. Observe that by using 
L'Hospital's rule, we can find value of $f$ for an infinitesimally small time step:
$$
\lim_{\Delta t \to 0} f = \lim_{\Delta t \to 0} \frac{ \frac{\norm{\ve{\omega}}}{2}
    \cos^{-2}\left(\frac{\norm{\ve{\omega}}\Delta t}{2}\right) }{ \norm{\ve{\omega}} } =
    \frac{1}{2}
$$
i.e.\ we obtain the original equation~\ref{quatRateOfChange} for the
instantaneous rate of change.

Now let $\Delta \q{q} = \Delta t \dot{\q{q}} =
    \frac{\Delta t}{2} \tilde{\ve{\omega}} \q{q}$.
From equation~\ref{quatRateOfChangeMagnitude} we find that
$\norm{\Delta\q{q}} = \frac{\norm{\ve{\omega}}\Delta t}{2}$.
Hence we can simplify the expression for the quaternion correcting factor by expressing it
in terms of $\Delta \q{q}$ as follows:
$$
\Delta t f \tilde{\ve{\omega}}\q{q} = \frac{\Delta t}{\norm{\ve{\omega}}\Delta t}
    \tan\left(\frac{\norm{\ve{\omega}}\Delta t}{2}\right) \tilde{\ve{\omega}} \q{q} =
    \tan\left(\norm{\Delta\q{q}}\right) \frac{\Delta\q{q}}{\norm{\Delta\q{q}}}
$$

This expression now has a clear geometric interpretation with respect to the 4D geometry:
$\norm{\Delta\q{q}}$ is measured in radians, and it corresponds to the \emph{correct}
angle between the old and the new vector $\q{q}$. Since $\q{q}$ and
$\dot{\q{q}}$ are orthogonal, we have a right-angled triangle between the origin,
the old and the new points $\q{q}$, and hence we can use the $\tan$ function to
evaluate the required length of the side in direction $\Delta\q{q}$.

Finally we can combine this correction and the subsequent quaternion normalisation into
a single function:
\begin{eqnarray*}
\q{q}(t+\Delta t) = \mathrm{Quergs}(\q{q}(t), \Delta\q{q}) &=&
    \frac{\q{q}(t) + \tan\left(\norm{\Delta\q{q}}\right)
        \frac{\Delta\q{q}}{\norm{\Delta\q{q}}}}{
    \norm{\q{q}(t) + \tan\left(\norm{\Delta\q{q}}\right)
        \frac{\Delta\q{q}}{\norm{\Delta\q{q}}}}} \\
&=& \frac{\q{q}(t) + \tan\left(\norm{\Delta\q{q}}\right)
        \frac{\Delta\q{q}}{\norm{\Delta\q{q}}}}{
    \sqrt{1 + \tan^2\left(\norm{\Delta\q{q}}\right)}} \\
&=& \left[\q{q}(t) + \tan\left(\norm{\Delta\q{q}}\right)
        \frac{\Delta\q{q}}{\norm{\Delta\q{q}}}\right]
    \cos\left(\norm{\Delta\q{q}}\right)
\end{eqnarray*}
The last expression is simplest (and again allows geometric interpretation), but probably
the penultimate expression is more useful for numerical evaluation, since it involves only
one trigonometric function.
