\documentclass{article}
\begin{document}
\title{Part II Project proposal (draft)}
\author{Martin Kleppmann (mk428)}
\maketitle

\section{Requirements}
The aim of this project is to develop and implement a simulation of movements
of the human body, in particular in interaction with another human. The
simulation is presented in 3-dimensional animated graphics with the intention
of looking fairly realistic.

The project divides into three parts, which are presented in the following
sections.

\subsection{Graphics engine}
The graphics engine should provide
\begin{itemize}
	\item an animatable 3D view of the scene, implemented using OpenGL.
	\item a 3D polygon mesh representing a human body, including bones and joints
        for the arms, hands, fingers, legs, back and neck, allowing it to be 
        deformed in a reasonably realistic-looking way.
  \item anatomical constraints on the possible angles of rotation at joints
        (comfortable and stretching range) and strength of muscles.
  \item an inverse kinematics solver to find the best bone positions for a
  		  given requirement (e.g.\ tip of index finger at a certain position).
  \item a collision detection algorithm for arbitrarily deformed meshes.
  \item {\em (optionally, if there is time)\/} facilities for facial animation.
\end{itemize}

\subsection{Human action}
The human action system includes
\begin{itemize}
  \item a physics simulation of gravity, collisions, inertia and friction,
  	    acting on the human model to prevent it from moving into
        impossible positions, and to allow actions like jumping in the air.
  \item a graphical method of programming movements or sequences of movements,
  		  and executing these on command, with interpolation between different
  		  positions of the body.
  \item a range of default movements, including a walk cycle.
  \item possibilities to merge automated behaviour (e.g.\ walking around while
        there is nothing else to do) with programmed behaviour.
  \item basic collision avoidance of obstacles.
  \item {\em (optionally)\/} an implementation of a pedestrian simulation model.
\end{itemize}

\subsection{Human interaction}
Human interaction takes place between two instances of the human model and
allows
\begin{itemize}
  \item programmed movements to touch a particular part (e.g.\ the right hand)
        of the other model's body with a particular part of one's own body,
        even if these parts are not at their usual positions (e.g.\ the arm is
        lifted).
  \item programmed reactions to particular actions of the other model (e.g.\
        dodging a punch).
  \item automatic, real-time matching of actions to the most appropriate
        programmed reactions.
  \item {\em (optionally)\/} two users to issue commands to the models and to
        view the scene concurrently over an IP network.
\end{itemize}


\section{Evaluation criteria}
Most of the features of the system can be demonstrated by sequences of
screenshots in the dissertation. To determine whether the behaviour exhibited
by the models can be deemed to be fairly realistic, the program will be
demonstrated to 10 students. After the demonstration they will be asked to
comment on the simulation in a brief questionnaire.

\section{References}
\begin{enumerate}
  \item Marco Gillies, Neil Dodgson: Behaviourally rich actions for
        user-controlled characters. Computers \& Graphics {\bf 28}~6,
        p.~945--954 (2004).
  \item Tony Polichroniadis, Neil Dodgson: Motion blending using a classifier
        system. Proceedings of the WSCG '99, p.~225--232.
  \item Michael Gleicher: Comparing Constraint-Based Motion Editing Methods.
  	    Graphical Methods {\bf 63}, p. 107-134 (2001).
\end{enumerate}

\end{document}
