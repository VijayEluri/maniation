\chapter{Notation}
\begin{tabular}{@{}lll@{}}
\renewcommand{\baselinestretch}{1.3}\small\normalsize
$a$ & italic Latin letter & scalar\\
$\theta$ & italic Greek letter & angle\\
\ve{a} & boldface Greek/Latin letter & column vector\\
\ve{0} & boldface figure 0 & null vector (of appropriate dimension)\\
\m{M} & sans serif upper-case letter & matrix\\
\m{1} & sans serif figure 1 & identity matrix (of appropriate dim.)\\
\q{q} & sans serif lower-case letter & quaternion\\
$\dot{c}$ & dot & first derivative with respect to time\\
$\ddot{c}$ & double dot & second derivative with respect to time\\
$\m{J}^T$ & superscript $T$ & matrix transpose\\
$\m{M}^{-1}$ & superscript $-1$ & matrix or quaternion inverse (eq.~\ref{quatInverse})\\
$\overline{\q{q}}$ & bar & quaternion conjugate (eq.~\ref{quatConjugate})\\
$\norm{\ve{a}}$ & norm & vector or quaternion magnitude (eq.~\ref{quatMagnitude})\\
$\tilde{\ve{a}}$ & tilde & corresponding quaternion (eq.~\ref{vectorToQuat})\\
$\dual{\ve{a}}$ & asterisk & dual tensor (eq.~\ref{dualTensor})\\
$\ve{a}\times\ve{b}$ & cross & vector cross product\\
$\ve{a}\cdot\ve{b}$ & dot & inner product\\
$\Re(\q{q})$ & real & real part of a quaternion\\
\end{tabular}
\vspace{10pt}

Given any vector $\ve{a} = (a_1, a_2, a_3)^T$, we define its dual
tensor, written as a $3\times3$ matrix, to be
\begin{equation}\label{dualTensor}
\dual{\ve{a}} = \left[\begin{array}{ccc}
    0 & -a_3 & a_2 \\ a_3 & 0 & -a_1 \\ -a_2 & a_1 & 0
    \end{array}\right]
\end{equation}
(see \cite{RHB:02,BaraffWitkin:97} and also Kalra~\cite{Kalra:95}, who defines it to be
the transpose of the expression above).
The dual allows us to rewrite a vector cross product as a matrix multiplication:
\begin{equation}
\ve{a}\times\ve{b} = \dual{\ve{a}}\,\ve{b}
\end{equation}
Note that $(\dual{\ve{a}})^T = -\dual{\ve{a}}$.

Let us also recall some basic identities of vector and matrix algebra~\cite{RHB:02}:
\begin{eqnarray*}
\ve{a}\times\ve{a} & = & \ve{0}\quad\mathrm{(the~null~vector)} \\
\ve{a}\times\ve{b} & = & -\ve{b}\times\ve{a} \\
\ve{a}\times(\ve{b} + \ve{c}) & = & \ve{a}\times\ve{b} + \ve{a}\times\ve{c} \\
\ve{a}\cdot\ve{b} & = & \ve{b}\cdot\ve{a} \\
\ve{a}\cdot\ve{b} & = & \ve{a}^T\,\ve{b} \\
\ve{a}\cdot(\ve{b} + \ve{c}) & = & \ve{a}\cdot\ve{b} + \ve{a}\cdot\ve{c} \\
\ve{a}\cdot(\ve{b}\times\ve{c}) & = & \ve{b}\cdot(\ve{c}\times\ve{a}) \\
    & = & \ve{c}\cdot(\ve{a}\times\ve{b}) \\
\ve{a}\times(\ve{b}\times\ve{c}) & = &
    \ve{b}(\ve{a}\cdot\ve{c}) - \ve{c}(\ve{a}\cdot\ve{b}) \\
(\m{A}\m{B})\m{C} & = & \m{A}(\m{B}\m{C}) \\
\m{A}(\m{B} + \m{C}) & = & \m{A}\m{B} + \m{A}\m{C} \\
(\m{A}\m{B})^T & = & \m{B}^T\m{A}^T \\
\m{A}\m{A}^{-1} = \m{A}^{-1}\m{A} & = & \m{1}\quad\mathrm{(the~identity~matrix)}
\end{eqnarray*}
