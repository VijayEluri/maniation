\section{A catalogue of constraint functions\label{constraintAppendix}}

Let us consider some examples of constraints to clarify the procedure.

\subsection{Fixed point in space (`Nail')\label{constrNail}}

This simple constraint `nails' a particular point in a rigid body to a fixed point
in world space. (It's a very flexible nail, because despite fixing the position, it allows all
three modes of rotation.) Let \ve{p} be the position of the centre of mass of our rigid
body, \ve{s} the vector from the centre of mass to the point in the body we want to attach,
$\ve{\omega}$ the angular velocity of the rigid body, and \ve{t} the coordinates in world
space that we want to nail the point to. Then we can set up a simple constraint function,
\begin{equation}
\ve{c} = \ve{p} + \ve{s} - \ve{t}
\end{equation}
which equals the null vector when $\ve{p}+\ve{s}$ and $\ve{t}$ coincide, as required.
Since this is a three-dimensional vector equation, we are actually defining three constraints
at once. \ve{t} does not change over time, so we obtain
\begin{eqnarray}
\dot{\ve{c}} &=& \dot{\ve{p}} + \ve{\omega}\times\ve{s} \nonumber\\
&=& \dot{\ve{p}} - \dual{\ve{s}}\,\ve{\omega} \\
\ddot{\ve{c}} &=& \ddot{\ve{p}} + \dot{\ve{\omega}}\times\ve{s} +
    \ve{\omega}\times(\ve{\omega}\times\ve{s}) \nonumber\\
&=& \ddot{\ve{p}} - \dual{\ve{s}}\,\dot{\ve{\omega}} -
    \dual{(\ve{\omega}\times\ve{s})}\,\ve{\omega}
\end{eqnarray}
(cf.\ similar derivations in~\cite{Kalra:95}). We have already moved the `chosen variables' to
the rightmost position of each product. We will now factor $\dot{\ve{c}}$ and write out the
components of $\m{J}$ in terms of the vector components:

\begin{equation}
\label{constrEx1J}
\dot{\ve{c}} = \left[\begin{array}{ccc} 1&0&0\\0&1&0\\0&0&1 \end{array}\right]
    \dot{\ve{p}} + \left[\begin{array}{ccc}
    0 & s_3 & -s_2 \\ -s_3 & 0 & s_1 \\ s_2 & -s_1 & 0
    \end{array}\right] \ve{\omega}
\end{equation}

The two matrices in equation~\ref{constrEx1J} thus form two slices of $\m{J}$ at
the locations appropriate for $\dot{\ve{p}}$ and $\ve{\omega}$. We now continue to the
next step of the procedure:

\begin{eqnarray}
\dot{\m{J}}\dot{\ve{x}} = 
\ddot{\ve{c}} - \m{J}\ddot{\ve{x}} &=&
    \big(\ddot{\ve{p}} - \dual{\ve{s}}\,\dot{\ve{\omega}} -
    \dual{(\ve{\omega}\times\ve{s})}\,\ve{\omega}\big) -
    (\ddot{\ve{p}} - \dual{\ve{s}}\,\dot{\ve{\omega}}) \nonumber\\
& = & -\dual{(\ve{\omega}\times\ve{s})}\,\ve{\omega} \nonumber\\
& = & \left[\begin{array}{ccc} 0 &
    \omega_1 s_2 - \omega_2 s_1 &
    \omega_1 s_3 - \omega_3 s_1 \\
    \omega_2 s_1 - \omega_1 s_2 & 0 &
    \omega_2 s_3 - \omega_3 s_2 \\
    \omega_3 s_1 - \omega_1 s_3 &
    \omega_3 s_2 - \omega_2 s_3 & 0
    \end{array}\right] \ve{\omega} \nonumber\\\label{constrEx1JDot}
\end{eqnarray}

We see that here there is only one slice for $\dot{\m{J}}$; the slice belonging to $\ddot{\ve{p}}$
is zero. Since $\ve{\omega}$ also occurs inside the matrix, there are actually several
alternative representations of this matrix which are equally valid.

\subsection{Ball-and-socket joint\label{constrJoint}}

Two rigid bodies are attached together at a particular point in each of the bodies.
They may not separate, but all three rotational degrees of freedom are permitted. This is
a good representation e.g.\ of a human shoulder joint.

Let $\ve{a}$ and $\ve{b}$ be the positions of the centres of mass in the first and
second rigid body respectively. Let $\ve{s}$ be the vector from $\ve{a}$ to the 
attachment point, and $\ve{t}$ the vector from $\ve{b}$ to the attachment point.
Also let $\ve{\omega}$ be the angular velocity of the first body, and $\ve{\phi}$ that
of the second. Then our constraint function and its derivatives are:

\begin{eqnarray}
\ve{c} &=& \ve{a} + \ve{s} - \ve{b} - \ve{t} \\
\dot{\ve{c}} &=& \dot{\ve{a}} + \ve{\omega}\times\ve{s} -
    \dot{\ve{b}} - \ve{\phi}\times\ve{t} \\
\ddot{\ve{c}} &=& \ddot{\ve{a}} + \dot{\ve{\omega}}\times\ve{s} +
    \ve{\omega}\times(\ve{\omega}\times\ve{s}) -
    \ddot{\ve{b}} - \dot{\ve{\phi}}\times\ve{t} -
    \ve{\phi}\times(\ve{\phi}\times\ve{t})
\end{eqnarray}

The rest of the derivation is very similar to the previous example. We obtain four
matrix slices for \m{J} and two slices for $\dot{\m{J}}$.


\subsection{Rotation axis restriction (`Joystick')\label{constrJoystick}}

We now have formulae to define a ball-and-socket joint. How can we express other types
of joints? A good way of doing this is by augmenting the ball-and-socket constraint with
additional constraints which restrict the set of valid rotations. In this section I
derive expressions for a constraint which prohibits rotation about one particular axis~--
or, in other words, confines the axis of any valid rotation to a plane. In engineering terms,
this is called a universal joint~\cite{Shabana:01}. Let us define a unit vector \ve{n} which
points in the direction of the axis we want to prohibit; equivalently, this is the normal of
the confinement plane.

It is not completely easy to visualize what this type of constraint means. One good way to look
at it is to consider a standard two-axis joystick. If it is placed on a table, the two axes of
rotation lie in a plane parallel to the surface of this table. But you cannot turn the stick about
its own axis. Hence the normal of the constraint plane is orthogonal to the table surface.
Don't be confused by the fact that the joystick handle happens to point in the direction of the
normal~-- any sort of obscure shape may be substituted in its place without changing the nature
of the constraint!

A more common sort of joint is the \emph{hinge} or \emph{revolute joint}, which we find in most
doors, in our knees and elbows. It allows rotation only about one particular axis. We can
conveniently express it by employing two `joystick' constraints on the same body, each of which
confines the axis to a plane. Provided the two planes are not parallel, the axis about which
rotation may occur is just the line of intersection of these two planes. In summary, to make a
revolute joint, we first add a ball-and-socket joint. Then we find two non-collinear vectors
which are both orthogonal to the hinge axis, and use them as normal vectors for two `joystick'
constraints. This reduces the original number of six degrees of freedom to one~-- the angle of
the hinge.

For this derivation I will use an alternative notation for quaternions, which is used by
Shoemake~\cite{Shoemake:85}, amongst others. Instead of using the complex constants \qi{}, \qj{}
and \qk{}, a quaternion is written as a pair consisting of a scalar (the real part) and a 3D
vector (the three imaginary parts):
\begin{equation}
\q{q} = q_w + q_x\qi + q_y\qj + q_z\qk = [q_w, (q_x, q_y, q_z)^T] = [q_w, \ve{q}_v]
\end{equation}
Using this notation, we can write the quaternion product in terms of vector dot and cross products:
\begin{equation}\label{quatProduct2}
\q{p}\q{q} = [p_w,\; \ve{p}_v]\;[q_w,\; \ve{q}_v] =
    [p_w q_w - \ve{p}_v \cdot \ve{q}_v,\;
    p_w \ve{q}_v + q_w \ve{p}_v + \ve{p}_v \times \ve{q}_v]
\end{equation}

We shall now consider the relative rotation of two rigid bodies. Say the first body has an
orientation quaternion \q{p} and angular velocity \ve{\phi}, and the second body orientation
\q{q} and angular velocity \ve{\omega}. Assume that each quaternion expresses the rotation
required to transform from the body's frame to the world frame. Then the quaternion product
$\q{p}^{-1}\q{q}$ is the rotation required to transform from the second body's frame to the first
one's~-- that is, the relative rotation of the two bodies.

To confine the axis of rotation, we use the fact that the axis is contained in the imaginary
parts of a quaternion (equation~\ref{quatRotation}, page~\pageref{quatRotation}). We want the dot
product of this axis and the normal vector \ve{n} to be zero. Conveniently, the dot product
happens to be implicitly present in the real part of the quaternion product
(see equation~\ref{quatProduct2}). Hence we can define our constraint
function as follows:
\begin{equation}\label{rotConstrEqn}
c = \Re(\tilde{\ve{n}}\, \q{p}^{-1}\q{q})
\end{equation}

As with complex numbers, the function $\Re$ returns only the real part of its argument.
Since the real part of $\tilde{\ve{n}}$ is zero by definition, this is just the dot product of
\ve{n} and the axis of $\q{p}^{-1}\q{q}$, with an extra minus sign in
front~(cf.\ equation~\ref{quatProduct2}). The constraint function is a scalar because
we are only losing one degree of freedom.

The derivative of \q{q} with respect to time is
$\dot{\q{q}} = \frac{1}{2}\tilde{\ve{\omega}}\q{q}$. Pushing the differential operator
onto the inside of a quaternion inverse produces a minus sign and reverses the order, provided
we are dealing with a unit quaternion:
\begin{equation}
\frac{\diff}{\diff t} \q{p} = \frac{1}{2}\tilde{\ve{\phi}}\,\q{p} \iff
    \frac{\diff}{\diff t} (\q{p}^{-1}) = -\frac{1}{2}\q{p}\,\tilde{\ve{\phi}}
\end{equation}

We now have everything in place to calculate the constraint function derivatives:
\begin{eqnarray}
\dot{c} & = & \label{constrJoystickC}
    \frac{1}{2} \Re(\tilde{\ve{n}}\, \q{p}^{-1}\, \tilde{\ve{\omega}}\,\q{q}) -
    \frac{1}{2} \Re(\tilde{\ve{n}}\, \q{p}^{-1}\, \tilde{\ve{\phi}}\, \q{q}) \\
\ddot{c} & = & \label{constrJoystickCDot}
    \frac{1}{2} \Re(\tilde{\ve{n}}\, \q{p}^{-1}\, \tilde{\dot{\ve{\omega}}}\,\q{q})
  - \frac{1}{2} \Re(\tilde{\ve{n}}\, \q{p}^{-1}\, \tilde{\dot{\ve{\phi}}}\, \q{q}) \\*
&&+ \frac{1}{4} \Re(\tilde{\ve{n}}\, \q{p}^{-1}\, \tilde{\ve{\omega}}\,\tilde{\ve{\omega}}\, \q{q})
  - \frac{1}{2} \Re(\tilde{\ve{n}}\, \q{p}^{-1}\, \tilde{\ve{\phi}}\,\tilde{\ve{\omega}}\, \q{q})
  + \frac{1}{4} \Re(\tilde{\ve{n}}\, \q{p}^{-1}\, \tilde{\ve{\phi}}\,\tilde{\ve{\phi}}\, \q{q})
    \nonumber
\end{eqnarray}

We manipulate these equations into the form required to find \m{J}:
\begin{eqnarray*}
\Re(\tilde{\ve{n}}\, \q{p}^{-1}\, \tilde{\ve{\omega}}\,\q{q}) & = &
    \Re\big( [0,\:\ve{n}]\: [p_w,\:-\ve{p}_v]\: [0,\:\ve{\omega}]\: [q_w,\:\ve{q}_v] \big) \\*
&=& \Re\big( [\ve{n}\cdot\ve{p}_v,\: p_w\ve{n} - \ve{n}\times\ve{p}_v]\:
             [-\ve{\omega}\cdot\ve{q}_v,\: q_w\ve{\omega} + \ve{\omega}\times\ve{q}_v] \big) \\*
&=& -(\ve{n}\cdot\ve{p}_v) (\ve{\omega}\cdot\ve{q}_v) -
    ( p_w\ve{n} - \ve{n}\times\ve{p}_v ) \cdot ( q_w\ve{\omega} + \ve{\omega}\times\ve{q}_v ) \\*
&=& -\ve{n}^T \ve{p}_v \ve{q}_v^T \ve{\omega} - ( p_w\ve{n} + \ve{p}_v\times\ve{n} )^T
    ( q_w\ve{\omega} - \ve{q}_v\times\ve{\omega} ) \\*
&=& -\ve{n}^T \ve{p}_v \ve{q}_v^T \ve{\omega} - ( p_w\ve{n}^T - \ve{n}^T\dual{\ve{p}_v} )
    ( q_w\ve{\omega} - \dual{\ve{q}_v}\,\ve{\omega} ) \\*
&=& -\ve{n}^T \big( \ve{p}_v \ve{q}_v^T + ( p_w\m{1} - \dual{\ve{p}_v} )
    ( q_w\m{1} - \dual{\ve{q}_v} ) \big)\, \ve{\omega}
\end{eqnarray*}

Here \m{1} denotes the $3\times3$ identity matrix. The same derivation is valid if we substitute
\ve{\phi} for \ve{\omega}, hence we obtain the Jacobian
\begin{eqnarray}
\m{J}\dot{\ve{x}} & = & 
    -\frac{1}{2} \ve{n}^T \big( \ve{p}_v \ve{q}_v^T + ( p_w\m{1} - \dual{\ve{p}_v} )
    ( q_w\m{1} - \dual{\ve{q}_v} ) \big)\, \ve{\omega} \nonumber\\*
&&  +\frac{1}{2} \ve{n}^T \big( \ve{p}_v \ve{q}_v^T + ( p_w\m{1} - \dual{\ve{p}_v} )
    ( q_w\m{1} - \dual{\ve{q}_v} ) \big)\, \ve{\phi}
\end{eqnarray}

Now the first two terms of equation~\ref{constrJoystickCDot} are generated by $\m{J}\ddot{x}$, so
for finding $\dot{\m{J}}$ we need only consider the last three terms. Let us evaluate the
penultimate term:
\begin{eqnarray*}
\Re(\tilde{\ve{n}}\, \q{p}^{-1}\, \tilde{\ve{\phi}}\,\tilde{\ve{\omega}}\, \q{q}) & = &
    \Re\big( [0,\:\ve{n}]\: [p_w,\:-\ve{p}_v]\: [0,\:\ve{\phi}]\: [0,\:\ve{\omega}]\:
    [q_w,\:\ve{q}_v] \big) \\*
&=& \Re\big( [0,\:\ve{n}]\: [\ve{p}_v \cdot \ve{\phi},\: p_w\ve{\phi} - \ve{p}_v\times\ve{\phi}]\:
    [-\ve{\omega}\cdot\ve{q}_v,\: q_w\ve{\omega} + \ve{\omega}\times\ve{q}_v] \big) \\*
&=& \Re\Big( \begin{array}[t]{lll} \big[ &
    \ve{n}\cdot(\ve{p}_v\times\ve{\phi}) - p_w \ve{\phi}\cdot\ve{n}, &\\ &
    (\ve{p}_v\cdot\ve{\phi}) \ve{n} + p_w\ve{n}\times\ve{\phi} -
        \ve{n}\times(\ve{p}_v\times\ve{\phi}) \quad\big] &\\
    \big[ & -\ve{\omega}\cdot\ve{q}_v,\: q_w\ve{\omega} +
        \ve{\omega}\times\ve{q}_v \quad\big] & \Big) \end{array} \\
&=& -\big( \ve{n}\cdot(\ve{p}_v\times\ve{\phi}) - p_w \ve{\phi}\cdot\ve{n} \big)
     \big( \ve{\omega}\cdot\ve{q}_v \big) \\* &&
    -\big( (\ve{p}_v\cdot\ve{\phi}) \ve{n} + p_w\ve{n}\times\ve{\phi} -
        \ve{n}\times(\ve{p}_v\times\ve{\phi}) \big) \cdot
     \big( q_w\ve{\omega} - \ve{q}_v\times\ve{\omega} \big) \\
&=& -\big( \ve{n}\cdot(\ve{p}_v\times\ve{\phi}) \big) \big( \ve{q}_v\cdot\ve{\omega} \big)
    + p_w (\ve{n}\cdot\ve{\phi}) (\ve{q}_v\cdot\ve{\omega}) \\* &&
    +\big( \ve{n}\cdot(\ve{q}_v\times\ve{\omega}) \big) \big( \ve{p}_v\cdot\ve{\phi} \big)
    - q_w (\ve{n}\cdot\ve{\omega}) (\ve{p}_v\cdot\ve{\phi}) \\* &&
    -\big( \ve{n}\times(p_w\ve{\phi} - \ve{p}_v\times\ve{\phi}) \big) \cdot
     \big( q_w\ve{\omega} - \ve{q}_v\times\ve{\omega} \big) \\
&=&  \ve{n}^T (p_w\m{1} - \dual{\ve{p}_v})\, \ve{\phi}\,\ve{q}_v^T \ve{\omega}
    -\ve{n}^T (q_w\m{1} - \dual{\ve{q}_v})\, \ve{\omega}\,\ve{p}_v^T \ve{\phi} \\* &&
    +(p_w\ve{\phi} - \ve{p}_v\times\ve{\phi})^T \dual{\ve{n}}
    (q_w\m{1} - \dual{\ve{q}_v})\, \ve{\omega}
\end{eqnarray*}

Fortunately, the other two terms of equation~\ref{constrJoystickCDot} we are interested in are
similar, so we can obtain them by substitution in the last expression. This gives us the
following expression for $\dot{\m{J}}$:

\begin{eqnarray}
\dot{\m{J}}\dot{\ve{x}} &=&
    \frac{1}{4}\Big( \ve{n}^T (p_w\m{1} - \dual{\ve{p}_v})\, \ve{\omega}\,\ve{q}_v^T
    -\ve{n}^T (q_w\m{1} - \dual{\ve{q}_v})\, \ve{\omega}\,\ve{p}_v^T \nonumber\\*&&\quad
    +(p_w\ve{\omega} - \ve{p}_v\times\ve{\omega})^T \dual{\ve{n}} (q_w\m{1} - \dual{\ve{q}_v})
    \nonumber\\*&&\quad
    -2\ve{n}^T (p_w\m{1} - \dual{\ve{p}_v})\, \ve{\phi}\,\ve{q}_v^T
    -2(p_w\ve{\phi} - \ve{p}_v\times\ve{\phi})^T \dual{\ve{n}} (q_w\m{1} - \dual{\ve{q}_v})
    \Big)\, \ve{\omega} + \nonumber\\*&&
    \frac{1}{4}\Big( \ve{n}^T (p_w\m{1} - \dual{\ve{p}_v})\, \ve{\phi}\,\ve{q}_v^T
    -\ve{n}^T (q_w\m{1} - \dual{\ve{q}_v})\, \ve{\phi}\,\ve{p}_v^T \nonumber\\*&&\quad
    +(p_w\ve{\phi} - \ve{p}_v\times\ve{\phi})^T \dual{\ve{n}} (q_w\m{1} - \dual{\ve{q}_v})
    \nonumber\\*&&\quad
    +2\ve{n}^T (q_w\m{1} - \dual{\ve{q}_v})\, \ve{\omega}\,\ve{p}_v^T \Big)\,\ve{\phi}
\end{eqnarray}


\subsection{Rotation angle limitation}

As mentioned in section~\ref{generalizedCollisions}, the constraint function of the last section
can be used in an inequality context to limit the angle of rotation rather than just the axes.
Adding a constant value to the constraint function makes no difference to its derivatives and
therefore leaves the Jacobians derived in the last section unchanged.

So the implementation is easy, but the challenge is to understand what effect such an inequality
actually has. Since quaternions are rather hard to visualize, let us translate the meaning of
equation~\ref{rotConstrEqn} into Euler angles. Choose three orthogonal axes with basis vectors
\ve{a}, \ve{b} and \ve{g} such that $\ve{a}\times\ve{b}=\ve{g}$ and $\ve{b}\times\ve{g}=\ve{a}$
and $\ve{g}\times\ve{a}=\ve{b}$ and $\norm{\ve{a}} = \norm{\ve{b}} = \norm{\ve{g}} = 1$.
Consider the rotation $\q{p}^{-1}\q{q}$ (the relative rotation between the two bodies), and
decompose it into a sequence of three rotations: first a rotation of $\alpha$ about the \ve{a}
axis, then a rotation of $\beta$ about the \ve{b} axis, and finally a rotation of $\gamma$ about
the \ve{g} axis. Thus we have
\begin{eqnarray}
\q{p}^{-1}\q{q} &=& \q{g}\q{b}\q{a} \\
\q{a} &=& \cos\big(\frac{\alpha}{2}\big) + \tilde{\ve{a}}\sin\big(\frac{\alpha}{2}\big) \\
\q{b} &=& \cos\big(\frac{\beta }{2}\big) + \tilde{\ve{b}}\sin\big(\frac{\beta }{2}\big) \\
\q{g} &=& \cos\big(\frac{\gamma}{2}\big) + \tilde{\ve{g}}\sin\big(\frac{\gamma}{2}\big)
\end{eqnarray}

Now if we evaluate the constraint function about each of the axes \ve{a}, \ve{b} and \ve{g}, we
get (skipping some boring algebra):
\begin{eqnarray}
\Re(\tilde{\ve{a}}\,\q{g}\q{b}\q{a}) &=& -\sin\big(\frac{\alpha}{2}\big)
     \cos\big(\frac{\beta }{2}\big)   \cos\big(\frac{\gamma}{2}\big)
    -\cos\big(\frac{\alpha}{2}\big)   \sin\big(\frac{\beta }{2}\big)
     \sin\big(\frac{\gamma}{2}\big)   \label{angleLimit1}\\
\Re(\tilde{\ve{b}}\,\q{g}\q{b}\q{a}) &=& -\cos\big(\frac{\alpha}{2}\big)
     \sin\big(\frac{\beta }{2}\big)   \cos\big(\frac{\gamma}{2}\big)
    -\sin\big(\frac{\alpha}{2}\big)   \cos\big(\frac{\beta }{2}\big)
     \sin\big(\frac{\gamma}{2}\big)   \label{angleLimit2}\\
\Re(\tilde{\ve{g}}\,\q{g}\q{b}\q{a}) &=& -\cos\big(\frac{\alpha}{2}\big)
     \cos\big(\frac{\beta }{2}\big)   \sin\big(\frac{\gamma}{2}\big)
    -\sin\big(\frac{\alpha}{2}\big)   \sin\big(\frac{\beta }{2}\big)
     \cos\big(\frac{\gamma}{2}\big)   \label{angleLimit3}
\end{eqnarray}

These expressions make clear how interdependent the three axes are~-- it is generally not
possible to change a constraint on one axis without affecting the others. The best way to proceed
from here is to enter inequalities using formulae~\ref{angleLimit1} to~\ref{angleLimit3} into
a program for graphical visualization, and to experimentally determine the values such that the
desired behaviour is achieved.


\subsection{Confinement to a plane (vertex/face collision) \label{vertexFaceConstraint}}

We want to define a constraint function whose value is the distance between a point
and a plane, where the point is attached to one rigid body, and the plane to another. The plane
is defined by a point in the plane and a normal vector. The distance should be positive if the
point is on the side of the plane pointed to by the normal, and negative if it is on the opposite
side. If we put this constraint directly into the Lagrange multiplier equation, it will enforce
the condition that the distance be zero~-- the point is confined to move only within the plane.
But if we use the same constraint function in an inequality, we have a handler for the
vertex/face collision case.

Say \ve{a} is the centre of mass position of the body to which the plane is attached, and \ve{s}
is the vector from the centre of mass to an arbitrary point in the plane. The angular velocity
of this body is \ve{\omega}. The plane has a unit normal vector $\hat{\ve{n}}$
($\norm{\hat{\ve{n}}} = 1$). The point we are interested in is $\ve{b} + \ve{t}$, where \ve{b} is
the centre of mass position of the body to which the point belongs. This body has angular
velocity \ve{\phi}.

Then the constraint function and its derivatives are given by
\begin{eqnarray}
c &=& (\ve{b} + \ve{t} - \ve{a} - \ve{s})\cdot\hat{\ve{n}} \\
\dot{c} &=& (\dot{\ve{b}} + \ve{\phi}\times\ve{t} - \dot{\ve{a}} - \ve{\omega}\times\ve{s})
    \cdot\hat{\ve{n}} + (\ve{b} + \ve{t} - \ve{a} - \ve{s})\cdot(\ve{\omega}\times\hat{\ve{n}})
    \nonumber\\
&=& \hat{\ve{n}}^T\dot{\ve{b}} - \hat{\ve{n}}^T\dot{\ve{a}} - \hat{\ve{n}}^T\dual{\ve{t}}\ve{\phi}
    + (\ve{a} - \ve{b} - \ve{t})^T \dual{\hat{\ve{n}}} \ve{\omega} \\
\ddot{c} &=& \big( \ddot{\ve{b}} + \dot{\ve{\phi}}\times\ve{t} +
    \ve{\phi}\times(\ve{\phi}\times\ve{t}) - \ddot{\ve{a}} \big) \cdot \hat{\ve{n}}+\nonumber\\*&&
    2\big(\dot{\ve{b}} + \ve{\phi}\times\ve{t} - \dot{\ve{a}}\big)
    \cdot \big(\ve{\omega}\times\hat{\ve{n}}\big)
    + \big(\ve{b} + \ve{t} - \ve{a}\big) \cdot \big( \dot{\ve{\omega}}\times\hat{\ve{n}} +
    \ve{\omega}\times(\ve{\omega}\times\hat{\ve{n}}) \big) \nonumber\\
&=&  \hat{\ve{n}}^T \ddot{\ve{b}}
    -\hat{\ve{n}}^T \ddot{\ve{a}}
    -\hat{\ve{n}}^T\dual{\ve{t}} \dot{\ve{\phi}}
    +(\ve{a} - \ve{b} - \ve{t})^T \dual{\hat{\ve{n}}} \dot{\ve{\omega}} + \\*&&
    \big( 2(\dot{\ve{a}} - \dot{\ve{b}} - \ve{\phi}\times\ve{t})^T \dual{\hat{\ve{n}}} 
    +(\ve{a} - \ve{b} - \ve{t})^T \dual{(\ve{\omega}\times\hat{\ve{n}})}\big) \ve{\omega}
    -\hat{\ve{n}}^T\dual{(\ve{\phi}\times\ve{t})} \ve{\phi} \nonumber
\end{eqnarray}
from which \m{J} and $\dot{\m{J}}$ can be read off as usual.


\subsection{Edge/edge collision\label{edgeEdgeConstraint}}

In this section we require a constraint function whose value is the shortest distance between
two straight lines. The problem is closely related to the one in
section~\ref{vertexFaceConstraint}. If used as an equality constraint, it could be used to
simulate two metal rods which are joined together such that the join can move up or down either
of the rods, but the rods always have to touch in one point~-- a kind of combination of a
ball-and-socket joint with two one-dimensional sliding rails. However, the practical use of such
a system is rather limited. Much more important is the use of this constraint as an inequality,
where it can handle the collision situation in which two edges collide.

We assume that each straight line is connected to a rigid body. The first body's centre of mass
is located at \ve{a}, its angular velocity is \ve{\omega}, \ve{s} is a vector from the centre
of mass to an arbitrary point on the line, and \ve{u} is a vector pointing along the line (unit
magnitude is not required). Similarly, the second body's CoM is \ve{b}, its angular velocity is
\ve{\phi}, \ve{t} points at the line and \ve{v} points in the line's direction. In this
derivation we shall assume that the lines are not parallel
($\norm{\ve{u}\times\ve{v}} \ne \ve{0}$); the parallel case is special and needs to be handled
separately.

If \ve{u} and \ve{v} are not parallel, we can find a unique plane which contains one line and is
parallel to the other. This plane has a normal vector $\ve{n} = \ve{u}\times\ve{v}$ (or
$\ve{n} = \ve{v}\times\ve{u}$). Interestingly this normal is the direction in which the force or
impulse acts in the event of a collision. It is not obvious that this is the case, but by playing
around with two books or similar it is possible to convince oneself. Then the closest distance
between the two lines is given by
\begin{equation}
c = (\ve{b} + \ve{t} - \ve{a} - \ve{s})\cdot\frac{\ve{n}}{\norm{\ve{n}}}
\end{equation}

The derivation of the Jacobian matrices involves some of the most messy linear algebra in this
project, so hold on tight. Let us first define a few auxiliary variables:
\begin{eqnarray}
\ve{n} & = & \ve{u}\times\ve{v} \\
h &=& \frac{1}{\norm{\ve{n}}} = (\ve{n}\cdot\ve{n})^{-\frac{1}{2}} \\
\ve{z} &=& \dot{\ve{n}} = \dot{\ve{u}}\times\ve{v} + \ve{u}\times\dot{\ve{v}} \nonumber\\*
    &=& (\ve{\omega}\times\ve{u})\times\ve{v} + \ve{u}\times(\ve{\phi}\times\ve{v}) \nonumber\\*
    &=& \dual{\ve{v}}\dual{\ve{u}}\ve{\omega} - \dual{\ve{u}}\dual{\ve{v}}\ve{\phi} \\
\ve{y} &=& \dot{\ve{z}} = \ddot{\ve{u}}\times\ve{v} + 2\,\dot{\ve{u}}\times\dot{\ve{v}} +
        \ve{u}\times\ddot{\ve{v}} \nonumber\\*
    &=& (\dot{\ve{\omega}}\times\ve{u})\times\ve{v} +
        (\ve{\omega}\times(\ve{\omega}\times\ve{u}))\times\ve{v} +
        2\,(\ve{\omega}\times\ve{u})\times(\ve{\phi}\times\ve{v}) + \nonumber\\*&&
        \ve{u}\times(\dot{\ve{\phi}}\times\ve{v}) +
        \ve{u}\times(\ve{\phi}\times(\ve{\phi}\times\ve{v})) \\*
    &=& \dual{\ve{v}}\dual{\ve{u}}\dot{\ve{\omega}} - \dual{\ve{u}}\dual{\ve{v}}\dot{\ve{\phi}} +
        \dual{\ve{v}}\dual{(\ve{\omega}\times\ve{u})}\ve{\omega} -
        \dual{\ve{u}}\dual{(\ve{\phi}\times\ve{v})}\ve{\phi} +
        2\,\dual{(\ve{\phi}\times\ve{v})}\dual{\ve{u}}\ve{\omega} \nonumber\\
\hat{\ve{n}} & = & h\ve{n} \\
\dot{\hat{\ve{n}}} & = & h\ve{z} - \frac{1}{2}\ve{n}(\ve{n}\cdot\ve{n})^{-\frac{3}{2}}
        (\ve{z}\cdot\ve{n} + \ve{n}\cdot\ve{z}) \nonumber\\*
    &=& h\ve{z} - h^3\,(\ve{n}\cdot\ve{z})\,\ve{n} \\
\ddot{\hat{\ve{n}}} & = & h\ve{y} - 2h^3 (\ve{n}\cdot\ve{z})\,\ve{z} -
        h^3 (\ve{n}\cdot\ve{y})\,\ve{n}
        -h^3 (\ve{z}\cdot\ve{z})\,\ve{n} + 3h^5 (\ve{n}\cdot\ve{z})^2\,\ve{n}
\end{eqnarray}

Now we can turn to calculating the derivatives of $c$:
\begin{eqnarray}
\dot{c} &=& \hat{\ve{n}}\cdot(\dot{\ve{b}} + \ve{\phi}\times\ve{t} - \dot{\ve{a}} -
        \ve{\omega}\times\ve{s}) + (\ve{b}+\ve{t}-\ve{a}-\ve{s})\cdot\dot{\hat{\ve{n}}} \\*
    &=& h\,(\ve{u}\times\ve{v})^T (\dot{\ve{b}} + \ve{\phi}\times\ve{t} - \dot{\ve{a}} -
        \ve{\omega}\times\ve{s}) + \nonumber\\*&&
        (\ve{b}+\ve{t}-\ve{a}-\ve{s})^T \big( h\ve{z} - h^3\,(\ve{n}\cdot\ve{z})\,\ve{n} \big)
        \nonumber\\
    &=& h\ve{u}^T\dual{\ve{v}}\dot{\ve{b}} - h\ve{u}^T\dual{\ve{v}}\dot{\ve{a}} -
        h\ve{u}^T\dual{\ve{v}}\dual{\ve{t}}\ve{\phi} +
        h\ve{u}^T\dual{\ve{v}}\dual{\ve{s}}\ve{\omega} + \nonumber\\*&&
        (\ve{b}+\ve{t}-\ve{a}-\ve{s})^T
        \big(h\dual{\ve{v}}\dual{\ve{u}}\ve{\omega} - h\dual{\ve{u}}\dual{\ve{v}}\ve{\phi} -
        h^3\dual{\ve{u}}\ve{v}\ve{u}^T\dual{\ve{v}}
        (\dual{\ve{v}}\dual{\ve{u}}\ve{\omega} - \dual{\ve{u}}\dual{\ve{v}}\ve{\phi})\big)
        \nonumber\\
    &=& h\ve{u}^T\dual{\ve{v}}\dot{\ve{b}} - h\ve{u}^T\dual{\ve{v}}\dot{\ve{a}} \nonumber\\*&&
        +\Big( h\ve{u}^T\dual{\ve{v}}\dual{\ve{s}} + h\,(\ve{b}+\ve{t}-\ve{a}-\ve{s})^T
        (\m{1} - h^2 \dual{\ve{u}}\ve{v}\ve{u}^T\dual{\ve{v}}) \dual{\ve{v}}\dual{\ve{u}}
        \Big)\:\ve{\omega} \nonumber\\*&&
        -\Big( h\ve{u}^T\dual{\ve{v}}\dual{\ve{t}} + h\,(\ve{b}+\ve{t}-\ve{a}-\ve{s})^T
        (\m{1} - h^2 \dual{\ve{u}}\ve{v}\ve{u}^T\dual{\ve{v}}) \dual{\ve{u}}\dual{\ve{v}}
        \Big)\:\ve{\phi} \\*
    &=& \m{J}\dot{\ve{x}} \nonumber\\
\ddot{c} &=& \hat{\ve{n}}\cdot\Big(\ddot{\ve{b}} + \dot{\ve{\phi}}\times\ve{t} +
        \ve{\phi}\times(\ve{\phi}\times\ve{t}) - \ddot{\ve{a}} - \dot{\ve{\omega}}\times\ve{s} -
        \ve{\omega}\times(\ve{\omega}\times\ve{s})\Big) + \nonumber\\*&&
        2\,\Big(\dot{\ve{b}} + \ve{\phi}\times\ve{t} - \dot{\ve{a}} - \ve{\omega}\times\ve{s}\Big)
        \cdot\dot{\hat{\ve{n}}} +
        \Big(\ve{b} + \ve{t} - \ve{a} - \ve{s}\Big)\cdot\ddot{\hat{\ve{n}}} \\
    &=& h\ve{u}^T\dual{\ve{v}}\ddot{\ve{b}} - h\ve{u}^T\dual{\ve{v}}\ddot{\ve{a}} -
        h\ve{u}^T\dual{\ve{v}}\dual{\ve{t}}\dot{\ve{\phi}} +
        h\ve{u}^T\dual{\ve{v}}\dual{\ve{s}}\dot{\ve{\omega}} + \nonumber\\*&&
        h\ve{u}^T\dual{\ve{v}}\dual{(\ve{\omega}\times\ve{s})}\ve{\omega} -
        h\ve{u}^T\dual{\ve{v}}\dual{(\ve{\phi}\times\ve{t})}\ve{\phi} \nonumber\\*&&
        +2h\,(\dot{\ve{b}} + \ve{\phi}\times\ve{t} - \dot{\ve{a}} - \ve{\omega}\times\ve{s})^T
        \Big(\dual{\ve{v}}\dual{\ve{u}}\ve{\omega} - \dual{\ve{u}}\dual{\ve{v}}\ve{\phi}
        \nonumber\\*&&\quad\quad -h^2\dual{\ve{u}}\ve{v}\ve{u}^T\dual{\ve{v}}
        (\dual{\ve{v}}\dual{\ve{u}}\ve{\omega} - \dual{\ve{u}}\dual{\ve{v}}\ve{\phi})\Big)
        \nonumber\\*&& +(\ve{b} + \ve{t} - \ve{a} - \ve{s})^T \Big(
        \nonumber\\*&&\quad\quad h\,\big(
        \dual{\ve{v}}\dual{\ve{u}}\dot{\ve{\omega}} - \dual{\ve{u}}\dual{\ve{v}}\dot{\ve{\phi}} +
        \dual{\ve{v}}\dual{(\ve{\omega}\times\ve{u})}\ve{\omega} \nonumber\\*&&\quad\quad\quad\quad
        -\dual{\ve{u}}\dual{(\ve{\phi}\times\ve{v})}\ve{\phi} +
        2\,\dual{(\ve{\phi}\times\ve{v})}\dual{\ve{u}}\ve{\omega} \big)
        \nonumber\\*&&\quad\quad
        -2h^3 (\dual{\ve{v}}\dual{\ve{u}}\ve{\omega} - \dual{\ve{u}}\dual{\ve{v}}\ve{\phi})
        \ve{u}^T\dual{\ve{v}}
        (\dual{\ve{v}}\dual{\ve{u}}\ve{\omega} - \dual{\ve{u}}\dual{\ve{v}}\ve{\phi})
        \nonumber\\*&&\quad\quad
        -h^3\dual{\ve{u}}\ve{v}\ve{u}^T\dual{\ve{v}} \big(
        \dual{\ve{v}}\dual{\ve{u}}\dot{\ve{\omega}} - \dual{\ve{u}}\dual{\ve{v}}\dot{\ve{\phi}} +
        \dual{\ve{v}}\dual{(\ve{\omega}\times\ve{u})}\ve{\omega} \nonumber\\*&&\quad\quad\quad\quad
        -\dual{\ve{u}}\dual{(\ve{\phi}\times\ve{v})}\ve{\phi} +
        2\,\dual{(\ve{\phi}\times\ve{v})}\dual{\ve{u}}\ve{\omega} \big)
        \nonumber\\*&&\quad\quad
        -h^3 \dual{\ve{u}}\ve{v}
        (\ve{\omega}^T\dual{\ve{u}}\dual{\ve{v}} - \ve{\phi}^T\dual{\ve{v}}\dual{\ve{u}})
        (\dual{\ve{v}}\dual{\ve{u}}\ve{\omega} - \dual{\ve{u}}\dual{\ve{v}}\ve{\phi})
        \nonumber\\*&&\quad\quad
        +3h^5 \dual{\ve{u}}\ve{v} \ve{u}^T\dual{\ve{v}}
        (\dual{\ve{v}}\dual{\ve{u}}\ve{\omega} - \dual{\ve{u}}\dual{\ve{v}}\ve{\phi})
        \ve{u}^T\dual{\ve{v}}
        (\dual{\ve{v}}\dual{\ve{u}}\ve{\omega} - \dual{\ve{u}}\dual{\ve{v}}\ve{\phi}) \Big) \nonumber
\end{eqnarray}
\newpage

Finally we subtract the terms generated by $\m{J}\ddot{\ve{x}}$ from $\ddot{c}$ and separate
into factors of \ve{\omega} and \ve{\phi} as usual:
\begin{eqnarray}
\dot{\m{J}}\dot{\ve{x}} &=& \bigg(
        h\ve{u}^T\dual{\ve{v}}\dual{(\ve{\omega}\times\ve{s})} \\*&&\quad\quad +
        2h\,(\dot{\ve{b}} + \ve{\phi}\times\ve{t} - \dot{\ve{a}} - \ve{\omega}\times\ve{s})^T
        (\m{1} - h^2\dual{\ve{u}}\ve{v}\ve{u}^T\dual{\ve{v}}) \dual{\ve{v}}\dual{\ve{u}}
        \nonumber\\*&&\quad\quad
        +(\ve{b} + \ve{t} - \ve{a} - \ve{s})^T \Big(
        h\dual{\ve{v}}\dual{(\ve{\omega}\times\ve{u})}
        +2h\,\dual{(\ve{\phi}\times\ve{v})}\dual{\ve{u}} \nonumber\\*&&\quad\quad\quad\quad
        -h^3\dual{\ve{u}}\ve{v}\ve{u}^T\dual{\ve{v}}\dual{\ve{v}}\dual{(\ve{\omega}\times\ve{u})}
        -2h^3\dual{\ve{u}}\ve{v}\ve{u}^T\dual{\ve{v}}\dual{(\ve{\phi}\times\ve{v})}\dual{\ve{u}}
        \nonumber\\*&&\quad\quad\quad\quad
        -2h^3\,(\dual{\ve{v}}\dual{\ve{u}}\ve{\omega} - \dual{\ve{u}}\dual{\ve{v}}\ve{\phi})
        \ve{u}^T\dual{\ve{v}}\dual{\ve{v}}\dual{\ve{u}}
        \nonumber\\*&&\quad\quad\quad\quad
        -h^3 \dual{\ve{u}}\ve{v} \big(
        \ve{\omega}^T\dual{\ve{u}}\dual{\ve{v}} - \ve{\phi}^T\dual{\ve{v}}\dual{\ve{u}}
        \nonumber\\*&&\quad\quad\quad\quad\quad\quad
        -3h^2 \ve{u}^T\dual{\ve{v}}
        (\dual{\ve{v}}\dual{\ve{u}}\ve{\omega} - \dual{\ve{u}}\dual{\ve{v}}\ve{\phi})
        \ve{u}^T\dual{\ve{v}} \big) \dual{\ve{v}}\dual{\ve{u}} \Big) \bigg)\: \ve{\omega}
        \nonumber\\*&&
        -\bigg(
        h\ve{u}^T\dual{\ve{v}}\dual{(\ve{\phi}\times\ve{t})} \nonumber\\*&&\quad\quad
        +2h\,(\dot{\ve{b}} + \ve{\phi}\times\ve{t} - \dot{\ve{a}} - \ve{\omega}\times\ve{s})^T
        (\m{1} - h^2\dual{\ve{u}}\ve{v}\ve{u}^T\dual{\ve{v}}) \dual{\ve{u}}\dual{\ve{v}}
        \nonumber\\*&&\quad\quad
        +(\ve{b} + \ve{t} - \ve{a} - \ve{s})^T \Big(
        h\dual{\ve{u}}\dual{(\ve{\phi}\times\ve{v})}
        -h^3\dual{\ve{u}}\ve{v}\ve{u}^T\dual{\ve{v}}\dual{\ve{u}}\dual{(\ve{\phi}\times\ve{v})}
        \nonumber\\*&&\quad\quad\quad\quad
        -2h^3\,(\dual{\ve{v}}\dual{\ve{u}}\ve{\omega} - \dual{\ve{u}}\dual{\ve{v}}\ve{\phi})
        \ve{u}^T\dual{\ve{v}} \dual{\ve{u}}\dual{\ve{v}}
        \nonumber\\*&&\quad\quad\quad\quad
        -h^3 \dual{\ve{u}}\ve{v} \big(
        \ve{\omega}^T\dual{\ve{u}}\dual{\ve{v}} - \ve{\phi}^T\dual{\ve{v}}\dual{\ve{u}}
        \nonumber\\*&&\quad\quad\quad\quad\quad\quad
        -3h^2 \ve{u}^T\dual{\ve{v}}
        (\dual{\ve{v}}\dual{\ve{u}}\ve{\omega} - \dual{\ve{u}}\dual{\ve{v}}\ve{\phi})
        \ve{u}^T\dual{\ve{v}}
        \big) \dual{\ve{u}}\dual{\ve{v}} \Big) \bigg)\: \ve{\phi} \nonumber
\end{eqnarray}
