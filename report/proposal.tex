%\documentclass{article}
%\begin{document}

{\hfill Martin Kleppmann}

{\hfill Corpus Christi College}

{\hfill mk428}

\vspace{3cm}
\centerline{\Large\bf Rigid body simulation for 3D character animation}
\vspace{1cm}
\centerline{19 October 2005}

\vspace{3cm}
\begin{tabular}{ll}
{\bf Project Originator:} & Martin Kleppmann \vspace{1.5cm}\\
{\bf Project Supervisor:} & Dr Neil Dodgson \vspace{0.5cm}\\
{\bf Signature:} & \vspace{1.5cm}\\
{\bf Director of Studies:} & Dr David Greaves \vspace{0.5cm}\\
{\bf Signature:} & \vspace{1.5cm}\\
{\bf Overseers:} & Dr Graham Titmus and Dr Markus Kuhn\\
\end{tabular}
\newpage

\section*{Project description}

The goal of this project is to develop a software tool for 3D character
animation, akin to (but simpler than) the applications used in the
production of cinematic special effects.

\subsection*{Core requirements}

The tool developed in this project should take as input a description
of a deformable polygon mesh bound to a skeleton. The skeleton should
contain anatomical constraints, in particular the range of angles each
joint may reach. The tool should calculate correct deformations of the
mesh for a given valid skeletal configuration and render the deformed
mesh to a screen.

In the next step, the tool should detect collisions within the
deformed mesh, and use the laws of classical mechanics to simulate a
rigid body, thus preventing different parts of the mesh from
intersecting. The simulated physical effects in a core implementation
should include collisions (impulse), friction forces, inertial mass
and gravitational forces. Additional information required for this
simulation (e.g.\ the mass of body attached to each bone of the
skeleton) should be included in the input file. Internal (muscular)
forces of the skeleton may also be specified as input, enabling the
character to move `out of its own strength'. All physical effects
should be simulated on a time-step basis.


\subsection*{Optional extensions}

Extensions to the program aim to take it further in the direction of
tools used by 3D animation artists to create realistic-looking
movements of humans or creatures. Possibilities include an inverse
kinematics solver, or interaction with other objects.


\subsection*{Acceptance criteria}

The program must correctly handle the animation of at least one mesh
resembling a human being. The mesh and skeleton need to include only
the main limbs (fingers, face etc. are considered optional). The
animation is correct if the character never enters an anatomically
impossible pose and does not violate the laws of physics.

Screenshots of several animations produced by the program will be
included in the dissertation. For further evaluation, other students
will be shown animations and asked to answer questions on whether they
believe the animation to be correct by the above definition.

Run-time performance is an optional acceptance criterion of this
project: ideally the program should run fast enough to calculate
animations on the example mesh in real time with fluent concurrent
graphical output, an average present-day PC provided. Such a
performance constraint may, however, require the use of very advanced
algorithms, and is therefore not regarded as part of the core project.


\subsection*{Non-Goals}

The following issues are \emph{not} objectives of this project:

\begin{itemize}
\item To achieve realistic graphical results. It is hard to objectively
  measure how realistic an animation looks, and a subjective judgement
  depends on many artistic factors. The graphics should be judged by
  `correctness' only.

\item To create an interactive user interface. The tool requires no user
  input beyond the mesh, skeleton, anatomy and animation files.

\item To achieve scientifically accurate calculations by using
  sophisticated numerical methods. Simple approximations are
  sufficient provided they do not lead to obviously wrong results.
\end{itemize}


\section*{Resources}

The main body of the program will be written in Java 1.5. This
language is well suited for the purpose because of its widespread
acceptance, good maintainability and the availability of extensive
libraries. The graphical output will be rendered using the Java3D
library.

Development will primarily be undertaken on my own PC, with a
revision-controlled copy on at least one other computer at any time,
and at least weekly backups on Pelican (daily when approaching a
deadline). I have tested all required software to ensure that it works
on PWF machines; in the case of a problem with my own PC, project
development could therefore seamlessly continue on the PWF.


\section*{Starting point}

I have some prior experience in basic 3D graphics programming using
OpenGL. I am not yet familiar with the Java3D library, and learning to
use it constitutes part of the preparation of the project.

For file input/output I intend to use an XML data binding tool which
I developed during the Summer 2005. Code automatically generated by
this tool will be included for marshalling/unmarshalling of data
structures, but the data binding tool is not itself part of this
project.

I do not intend to make use of any external or previously written code
apart from this tool and the standard Java runtime libraries.
The physics simulation will be based primarily on my knowledge from
Part~IA Natural Sciences and additional reading if necessary. For the
collision detection part of the project, I will search for literature
on appropriate algorithms during the preparation phase.

I would like to create example 3D meshes myself using the Blender
modelling software. I already have a moderate amount of experience
in using this application.


\section*{Timetable}

The project timetable aims to complete long before the final deadline
in order to leave additional time for exam revision. It is divided
into 2-week blocks.

\subsubsection*{10 Oct -- 23 Oct:}
Create 3D model of a human, including skeleton. Write project proposal,
get approval of supervisor and overseers. Revise Part~IA physics.

Milestones: A simple polygon mesh resembling a human; project proposal
submitted.

\subsubsection*{24 Oct -- 6 Nov:}
Plan the underlying data structures. Describe these in an XML schema.
Create parser and writer for mesh file format using data binding.
Render the undeformed mesh using Java3D. Work out the maths behind
mesh deformation and physics.

Milestones: Working program which can read in the human mesh from a
file and display it on screen. Sketch of the calculations underlying
the physical simulation.

\subsubsection*{7 Nov -- 20 Nov:}
Survey of literature on collision detection algorithms. Select an
appropriate algorithm and plan the required data structures.
Implement mesh deformation as planned previously. Prepare sample
animations.

Milestone: The working program is extended to display an animated
deformation of the model on screen.

\subsubsection*{21 Nov -- 4 Dec:}
Implement an appropriate collision detection algorithm as described
in literature. Perform some preliminary performance measurements to
estimate the efficiency of the algorithm. If necessary, try to
optimize the data structure to require less processing.

Milestone: The working program is extended to produce some form of
output whenever a collision is detected.

\subsubsection*{5 Dec -- 18 Dec:}
Implement the physical model of forces, collisions and friction. Test
the program at different deformations, ensure that the collisions
prevent the mesh from intersecting.

Milestone: The output of the working program should now look a bit
like a person in zero gravity (except for the missing inertial mass).

\subsubsection*{19 Dec -- 1 Jan:}
Holiday.

\subsubsection*{2 Jan -- 15 Jan:}
Implement the remaining physical model, in particular inertia and
gravity. Give the model a sense of balance, making it stand upright.

Milestone: The core requirements of the project should now be
fulfilled (except for some potential inaccuracies and inefficiencies).

\subsubsection*{16 Jan -- 29 Jan:}
Time to investigate and implement an extension.

\subsubsection*{30 Jan -- 12 Feb:}
Clean up the implementation, fix remaining bugs. Start thorough
evaluation: Create demo animations, demonstrate to test users and
collect feedback.

Milestones: Screenshots of the working program. Progress report.

\subsubsection*{13 Feb -- 26 Feb:}
Write outline of the whole dissertation. Refine the introduction and
preparation chapters. Create diagrams and illustrations as necessary.

Milestone: First two chapters of the dissertation completed.

\subsubsection*{28 Feb -- 12 Mar:}
Write detail of the implementation, evaluation and conclusions
chapters. Ask other people to read draft dissertation.

Milestone: Dissertation completely written.

\subsubsection*{13 Mar -- 19 Mar (1 week):}
Finalize dissertation, refinement of language. Prepare for printing.

%\end{document}
