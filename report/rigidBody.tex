\section{Free precession\label{correctBrettAppendix}}

This argument is modelled after~\cite{Ruf:02}. The moment of inertia $\ve{L}$ of a rigid
body is defined as
\begin{equation}
\label{correctBrett1}
\ve{L} = \m{I}\ve{\omega}
\end{equation}
where $\m{I}$ is the inertia tensor and $\ve{\omega}$ is the angular velocity vector.
Torque is the rate of change of angular momentum over time:
\begin{equation}
\label{correctBrett2}
\ve{\tau} = \dot{\ve{L}} = \dot{\m{I}}\ve{\omega} + \m{I}\dot{\ve{\omega}}
\end{equation}

We can further evaluate $\dot{\m{I}}$ by writing \m{I} as a product with some rotation matrix
$\m{R}$ and its transpose:
\begin{equation}
\label{correctBrett3}
\m{I} = \m{R}\m{I}_\mathrm{body}\m{R}^T
\end{equation}

It can be shown that such a decomposition of the inertia tensor always exists, and that
$\m{I}_\mathrm{body}$ is a diagonal, time-invariant matrix containing the moments
of inertia about the body's principal axes~\cite{Feynman:63}. Hence we obtain
\begin{equation}
\label{correctBrett4}
\dot{\m{I}} = \dot{\m{R}}\m{I}_\mathrm{body}\m{R}^T +
    \m{R}\m{I}_\mathrm{body}\frac{\diff}{\diff t}\m{R}^T
\end{equation}

Witkin~\cite{BaraffWitkin:97} derives that $\dot{\m{R}} = \dual{\ve{\omega}}\,\m{R}$
for a rotation matrix $\m{R}$ and an angular velocity vector $\ve{\omega}$.
Using this identity and taking the differential operator onto the inside of the
transpose at the end of equation~\ref{correctBrett4},
\begin{eqnarray}
\dot{\m{I}} &=& \dual{\ve{\omega}}\,\m{R}\m{I}_\mathrm{body}\m{R}^T +
    \m{R}\m{I}_\mathrm{body}(\dual{\ve{\omega}}\,\m{R})^T \nonumber\\
&=& \dual{\ve{\omega}}\,\m{R}\m{I}_\mathrm{body}\m{R}^T -
    \m{R}\m{I}_\mathrm{body}\m{R}^T\dual{\ve{\omega}} \nonumber\\
&=& \dual{\ve{\omega}}\,\m{I} - \m{I}\dual{\ve{\omega}} \label{correctBrett5}
\end{eqnarray}

Substituting equation~\ref{correctBrett5} back into~\ref{correctBrett2}:
\begin{eqnarray}
\ve{\tau} & = & \m{I}\dot{\ve{\omega}} + \dual{\ve{\omega}}\,\m{I}\ve{\omega} -
    \m{I}\dual{\ve{\omega}}\,\ve{\omega} \nonumber \\
& = & \m{I}\dot{\ve{\omega}} + \dual{\ve{\omega}}\,\m{I}\ve{\omega} \label{correctBrett6}
\end{eqnarray}

Equation~\ref{correctBrett6} corrects the similar expression in~\cite{Saunders:PhD},
page~34. This means that the angular acceleration of a rigid body is given by
\begin{equation}
\label{correctBrett7}
\dot{\ve{\omega}} = \m{I}^{-1} (\ve{\tau} - \dual{\ve{\omega}}\,\m{I}\ve{\omega}).
\end{equation}
where \ve{\tau} is the sum of all torque vectors acting on the body. This means that even if
there are no torques, its angular velocity may change if \m{I} is not diagonal (i.e.\ if
the body is somehow asymmetric). This effect is called \emph{free precession}.

In a simulation, we usually integrate over torques to find angular momentum, and then calculate
the angular velocity from the momentum in each time step using the current moment of inertia. In
this case, the angular acceleration in equation~\ref{correctBrett7} is not needed. It is required
only for purposes of computing constraint forces and torques.
