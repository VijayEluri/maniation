\section{Constrained rigid body dynamics}
The methods we have developed so far allow the simulation of a single rigid body with forces
acting on it. Further challenges arise when we consider multibody systems in which the bodies
interact in some way. We will primarily be interested in articulated bodies, that is, systems
of individual rigid bodies held together by joints. Joints allow rotation but cannot be
separated. (The human body is a good example of such an articulated body.)

There are various strategies for implementing articulated bodies. {\em Lagrangian mechanics}
is a commonplace method in physics and engineering, in which the state of the system is expressed
in terms of a set of {\em generalised coordinates}. The position and orientation of each sub-body
is then a function of these generalised coordinates, and outside forces can be transformed
back into the generalised coordinates, where they are handled. For example, consider two rigid
bodies which are held together by a ball-and-socket joint~\cite{Kalra:95}. Without the joint,
the system would have 12 degrees of freedom at rest: three translational and three rotational
degrees for each of the bodies. In a Lagrangian formulation, the two connected bodies would be
expressed in nine coordinates: six for translation and rotation of the first body, and three
for the rotation of the second body with respect to the first.

Lagrangian mechanics has the advantage that if formulated correctly, the system will always in a
legal state, no matter what the values of the coordinates are. It also allows very fast
computation. On the downside, the functions to transform the coordinates become very hard or
impossible to find for complicated systems, so the method does not scale well.

An alternative approach to the problem is to initially grant each sub-body its full number of
degrees of freedom, and then to impose constraints on the system. We can then ensure that the
constraints are always maintained by excerting appropriate forces and torques on the bodies.
Finding these forces is commonly done by means of {\em Lagrange multipliers}~-- note that
despite the similarity in name, this method has very little in common with Lagrangian mechanics.

Lagrange multipliers are used to implement constrained rigid body mechanics in this project,
because once set up, such a system can handle a large variety of situations with ease.
The mathematical formulation of Lagrange multipliers is derived in~\cite{BaraffWitkin:97} and
extended in~\cite{Saunders:PhD} (also see~\cite{Baraff:96} for an optimized algorithm),
therefore we only state the results here.

Consider a system of $n$ rigid bodies at a particular point in time. Each body has a vector
pointing to its centre of mass, and an orientation. Assume that we can concatenate these
parameters of all bodies into a global state vector\footnote{Witkin calls this vector
$\ve{q}$; we use $\ve{x}$ here to avoid confusion with quaternions.} $\ve{x}$.
Containing a value for each degree of freedom, $\ve{x}$ will have $6n$ rows. In practice
we will never need to write down $\ve{x}$ explicitly; instead we will work with its
first and second derivative in time, $\dot{\ve{x}}$ and $\ddot{\ve{x}}$ respectively.
Say a body $i$ has a centre of mass position $\ve{r}_i$ and an angular velocity
$\ve{\omega}_i$. Then we would have
\begin{equation}
\label{lagrangeStateVector}
\dot{\ve{x}} = \left[ \begin{array}{c}
    \dot{\ve{r}}_1\\ \ve{\omega}_1\\ \vdots\\ \dot{\ve{r}}_n\\ \ve{\omega}_n \end{array}\right]
\quad\quad\mathrm{and}\quad\quad
\ddot{\ve{x}} = \left[ \begin{array}{c}
    \ddot{\ve{r}}_1\\ \dot{\ve{\omega}}_1\\ \vdots\\ \ddot{\ve{r}}_n\\ \dot{\ve{\omega}}_n
    \end{array}\right]
\end{equation}

Now assume that we want to impose $m$ constraints on this system, where each constraint
effectively reduces the number of degrees of freedom by one. (A constraint specifying a
ball-and-socket joint, as described earlier, would therefore have to be expressed as a
vector of three constraints.) Express each constraint as a function $c$ which is zero
when the constraint is satisfied. We can then combine all constraint functions into a
single $m$-row constraint vector $\ve{c}$. Each valid configuration of the system must
satisfy $\ve{c} = \ve{0}$, the null vector.

To find the constraint maintaining forces, we must know how a change in state or a change in
time affects the value of $\ve{c}$. To this end, we must calculate the first two partial
derivatives of $\ve{c}$ with respect to time, $\dot{\ve{c}}$ and $\ddot{\ve{c}}$.
We also calculate the Jacobian matrix~\cite{RHB:02} $\m{J}$ which contains all partial
derivatives of $\ve{c}$ w.r.t. all coordinates of $\ve{c}$, and the similarly
defined Jacobian $\dot{\m{J}}$ of $\dot{\ve{c}}$:
\begin{equation}
J_{ij} = \frac{\partial c_i}{\partial x_j} \quad\quad\mathrm{and}\quad\quad
\dot{J}_{ij} = \frac{\partial \dot{c}_i}{\partial x_j}
\end{equation}

Both $\m{J}$ and $\dot{\m{J}}$ are $6n\times m$ matrices. However, since we do not
have an explicit expression for $\ve{x}$, we cannot calculate them directly using
partial differentiation. Instead we make use of the following results of the chain rule:
\begin{equation}
\label{cDotAndCDDot}
\dot{\ve{c}} = \m{J}\dot{\ve{x}} \quad\quad\mathrm{and}\quad\quad
\ddot{\ve{c}} = \dot{\m{J}}\dot{\ve{x}} + \m{J}\ddot{\ve{x}}.
\end{equation}
Appendix~\ref{constraintAppendix} gives examples of derivations using these equations.

Two more quantities are required before we can determine the forces and torques which maintain
our constraints. We need to add up all the other forces and torques acting on each of the $n$
sub-bodies (for example due to gravity or muscular activity) and concatenate them into a single
$6n$-row vector \ve{\Phi}, just as we did for \ve{x} (equation~\ref{lagrangeStateVector}):
\begin{equation}
\ve{\Phi} = \left[\begin{array}{l}
    \ve{F}_1 \\ \ve{\tau}_1 \\ \vdots \\ \ve{F}_n \\ \ve{\tau}_n
    \end{array}\right]
\end{equation}
where $\ve{F}_i$ denotes the sum of all forces acting on the centre of mass of body~$i$, and
$\ve{\tau}_i$ is the sum of all torque vectors acting on body~$i$.

And finally, we require the {\em mass-inertia tensor} \m{M} and its inverse
$\m{M}^{-1}$. \m{M} is a $6n\times6n$ matrix of the form
\begin{equation}
\label{massInertiaTensor}
\m{M} = \left[ \begin{array}{ccccccc}
    \mu_1\m{1} \\ & \m{I}_1 \\ &&
    \mu_2\m{1} \\ &&& \m{I}_2 \\ &&&& \ddots \\ &&&&&
    \mu_n\m{1} \\ &&&&&& \m{I}_n
    \end{array}\right]
\end{equation}

Here $\mu_i$ denotes the (scalar) mass of body $i$, \m{1} stands for the $3\times3$
identity matrix, and $\m{I}_i$ is the inertia tensor (written as a $3\times3$ matrix)
of body $i$ in the world frame. All empty regions are zero. While $\mu_i$ will typically
stay constant over time, $\m{I}_i$ may depend on the orientation of body $i$.\footnote{If
we know the principal axes of the body, we can express \m{I} in a time-invariant
form combined with rotations into the principal axes frame and back again
(see~\cite{BaraffWitkin:97}), which saves us a lot of effort.} The inverse $\m{M}^{-1}$
has a very similar form: it is obtained by replacing each $\mu$ by $\frac{1}{\mu}$ and
each $\m{I}$ by $\m{I}^{-1}$ in equation~\ref{massInertiaTensor}.

Now we can set up the equation\footnote{Here we adopt the sign convention
of~\cite{BaraffWitkin:97}, which is the opposite of~\cite{Saunders:PhD}.}
\begin{equation}
\label{lagrangeEquation}
-\m{J}\m{M}^{-1}\m{J}^T\ve{\lambda} = \dot{\m{J}}\dot{\ve{x}} +
    \m{J}\m{M}^{-1}\ve{\Phi} + k\ve{c} + d\dot{\ve{c}}
\end{equation}
in which all variables except $\ve{\lambda}$ are known. The meaning of this equation and
its two scalar constants $k$ and $d$ is discussed in~\cite{BaraffWitkin:97}
and~\cite{Saunders:PhD}. For us it will suffice to consider it to be a `black box' which can be
given to a linear equation solver to obtain $\ve{\lambda}$.

$\ve{\lambda}$ is an $m$-row vector of so-called Lagrange multipliers (after which this method
is named), where $m$ is the number of constraints. Once we have solved for it, we can compute
the expression
\begin{equation}
\hat{\ve{\Phi}} = \m{J}^T\,\ve{\lambda}.
\end{equation}
$\hat{\ve{\Phi}}$ has the same format as $\ve{\Phi}$, and indeed it is the vector containing
the forces and torques which we need to additionally apply to each body to ensure that they
continue to satisfy the constraints! All we therefore need to do is to add
$\ve{\Phi} + \hat{\ve{\Phi}}$ and to feed the result into our ODE solver.
