\section{Collision and contact}

We now have all the mathematical facilities we need to simulate mechanical systems in which
the acceleration functions are continuous over time. This already covers a large number of
interesting systems, but unfortunately excludes any sort of collision between bodies. If we want
to examine systems of separable but non-penetrating bodies, we have to open a whole new can of
worms.

Let us distinguish two cases of contact between bodies: resting contact and colliding contact.
A book lying on your desk is in resting contact with the surface: their relative velocity is zero,
but the desk excerts a force on the book to prevent it from penetrating into the surface.
Colliding contact is given for example between the ground and a ball bouncing off it. In real
life, the ball stays in contact with the ground for a finite length of time, during which it is
deformed and experiences a finite force accelerating it upwards. In our simulation, however, we
are working with rigid bodies which cannot be deformed, so we want the collision to happen
instantaneously in the moment when the ball touches the ground. One way of looking at this is as
an infinitely strong force acting over an infinitely short time, but a more useful representation
is in terms of an impulse exchanged between the bodies.

Creating a simulation involving contacts is generally seen as a difficult task.
Baraff~\cite{BaraffWitkin:97} derives equations to handle collisions, and outlines (with a number
of errors) a way of handling resting contact. Unfortunately his collision handling does not work
in combination with constraints~-- it violates the assumption that the constraint function is
twice differentiable~-- and his resting contact computation relies on a complicated and uncommon
numerical routine. In this project, a new method of handling contacts is developed, which is
simpler to implement, more powerful, and works perfectly together with constraints like those
described in section~\ref{constraints}.

In the rest of this section we will examine the simulation only at one point in time, at which
the bodies in question are in contact. We also assume that we know the place at which the contact
occurs. Algorithms for finding both the time and place of contact are discussed elsewhere in this
report.

\subsection{Resting contact}

To keep things manageable, let us assume that we are dealing only with polyhedral bodies~-- bodies
whose exterior consists of flat polygon faces, delimited by vertices, joined by straight edges.
Since we will actually be working with a triangle mesh, this is fine.

There are two main ways in which polyhedra can be in contact: either by a vertex of one body being
inside a face of the other body, or by an edge of one body intersecting an edge of the other.
There remain a few corner cases, but we will think about those later. The book lying on your desk
may for example have four vertex/face contacts, each of the four corners of its back cover
touching the face of the desk surface. You may visualize edge/edge contacts by holding the book
such that its spine is touching the front edge of your desk.

In both cases, for the bodies not to interpenetrate, they must excert equal and opposite forces on
each other. This force must exactly balance the outside force, otherwise they would accelerate
apart. Also in both cases the direction of the force is perpendicular to the plane of contact. In
the vertex/face case, this plane is just the plane of the face polygon. In the edge/edge case, the
plane goes through the point of contact and is spanned by the directions of the two edges involved.
Modulo its sign, this plane's unit normal vector is unique~-- except if the two edges are
parallel, which is a case we will ignore for now.


