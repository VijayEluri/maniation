\section{Quaternions}
\subsection{The need for Quaternions}
Besides its position, every rigid body in 3D space may have an orientation. This introduces an
additional three degrees of freedom for each stationary body (six degrees of freedom if
angular velocity is included). Unfortunately, there is no similarly neat representation of
orientation as cartesian coordinates are for the position. The most common schemes describe
orientation in terms of a rotation operation which transforms a vector in the body's local
coordinates into world coordinates (or vice-versa). There is, however, no obvious best
answer to the question how to represent this rotation: the three most common representations,
described here, all have their advantages and disadvantages.

{\em Euler angles} are probably the most intuitive representation of a 3D rotation, describing
it as a series of three rotations about different axes. These axes are fixed by convention, so it
suffices to specify the three angles of rotation. However, this scheme has a number of drawbacks
which are extensively discussed in the literature: amongst other things, it is possible that
rotation about an axis freezes during an animation (``Gimbal lock'') unless all special cases
are handled very carefully.

{\em Rotation matrices} are commonly used because they are well understood, easily generalize
to other dimensions and allow efficient combination with other linear transformations (scaling
and shearing~-- translation may also be included if homogeneous coordinates are employed).
Unfortunately, many operations required for animation of rotations are awkward to implement
since this representation uses nine numbers (a $3\times3$ matrix) to represent three degrees
of freedom, thus introducing six additional side-conditions.

{\em Quaternions} are a popular alternative to the two previous schemes, and they are also
used extensively in this project.

\subsection{Definition and properties}
Mathematically, quaternions can be regarded as numbers with one real part and three
distinct imaginary parts:
\begin{equation}
\mathbf{q} = w + x\mathbf{i} + y\mathbf{j} + z\mathbf{k}
\end{equation}
where $w$, $x$, $y$ and $z$ are real and $\mathbf{i}$, $\mathbf{j}$ and $\mathbf{k}$ satisfy
\begin{equation}
\mathbf{i}^2 = \mathbf{j}^2 = \mathbf{k}^2 = \mathbf{i}\mathbf{j}\mathbf{k} = -1.
\end{equation}
From this follows that
$\mathbf{i}\mathbf{j} = -\mathbf{j}\mathbf{i} = \mathbf{k}$ and
$\mathbf{j}\mathbf{k} = -\mathbf{k}\mathbf{j} = \mathbf{i}$ and
$\mathbf{k}\mathbf{i} = -\mathbf{i}\mathbf{k} = \mathbf{j}$.
Note that multiplication is not commutative.

We will also need the conjugate and the inverse of a quaternion. In analogy to
complex numbers, these are given respectively by
\begin{eqnarray}
\mathbf{q}^\ast & = & w - x\mathbf{i} - y\mathbf{j} - z\mathbf{k}\\
\mathbf{q}^{-1} & = & \frac{\mathbf{q}^\ast}{||\mathbf{q}||^2}
\end{eqnarray}

On the other hand, as there is a similarity to 4D vectors, we also define a
quaternion dot product as
\begin{equation}
\mathbf{q}_1\cdot\mathbf{q}_2 =
    (w_1 + x_1\mathbf{i} + y_1\mathbf{j} + z_1\mathbf{k})\cdot
    (w_2 + x_2\mathbf{i} + y_2\mathbf{j} + z_2\mathbf{k}) =
    w_1 w_2 + x_1 x_2 + y_1 y_2 + z_1 z_2
\end{equation}
The dot product is commutative, contrary to the `normal' quaternion product.

Sometimes we will need to relate a 3D vector to a quaternion with zero real part.
For a given vector $\mathbf{u} = (u_1, u_2, u_3)$ we define the corresponding
quaternion to be
\begin{equation}
\label{vectorToQuat}
\tilde{\mathbf{u}} = u_1\mathbf{i} + u_2\mathbf{j} + u_3\mathbf{k}.
\end{equation}

The complex constants $\mathbf{i}$, $\mathbf{j}$ and $\mathbf{k}$ are required for the algebra
only, therefore we can represent a quaternion as four numbers $(w,x,y,z)$. It turns out that
quaternions neatly represent a rotation in 3D space (similarly to the way that ordinary complex
numbers represent a rotation in the 2D Argand diagram) if we require their magnitude to be
equal to one:
\begin{equation}
\label{unitQuat}
||\mathbf{q}||^2 = w^2 + x^2 + y^2 + z^2 = 1.
\end{equation}
This condition reduces the number of degrees of freedom to three, as required.

Every unit quaternion represents a rotation of angle $\theta$ about an arbitrary axis.
If the axis passes through the origin and has a direction given by the vector
$\mathbf{a} = (a_1, a_2, a_3)$ with $||\mathbf{a}|| = 1$, the quaternion describing this rotation is
\begin{equation}
\label{quatRotation}
\mathbf{q} = \cos\left(\frac{\theta}{2}\right) + (a_1\mathbf{i} + a_2\mathbf{j} +
    a_3\mathbf{k}) \sin\left(\frac{\theta}{2}\right).
\end{equation}
It can easily be verified that this quaternion always has unit magnitude.
We shall assume throughout this project that the rotation thus described is clockwise (as seen
when looking in the direction of the vector $\mathbf{a}$), i.e. that it is given by the
``right-hand rule''.

To rotate a vector $\mathbf{v} = (v_1, v_2, v_3)$ by a quaternion $\mathbf{q}$, we first
convert it into its corresponding quaternion $\tilde{\mathbf{v}}$ as defined in
equation~\ref{vectorToQuat} and then calculate the quaternion product
\begin{equation}
\label{quatTransform}
\tilde{\mathbf{v}}^\prime = \mathbf{q}\tilde{\mathbf{v}}\mathbf{q}^{-1}
\end{equation}

If we expand this formula, we find that the real part of the result is always zero, and
that the rotated vector $\mathbf{v}^\prime = (v_1^\prime, v_2^\prime, v_3^\prime)$ is
the vector corresponding to $\tilde{\mathbf{v}}^\prime$ (i.e. $\mathbf{v}^\prime$ is
contained in the three complex parts of the quaternion product).

Some authors, notably Shoemake, choose to define the product in equation~\ref{quatTransform} in
the reverse order. The choice is a matter of convention, since it merely changes the effect
of this operation from being a clockwise to a counter-clockwise rotation. We chose the clockwise
convention because it is consistent with the usual definition of the angular velocity vector
in physics.

Observe that under this convention, if $\mathbf{q}$ is itself a product of quaternions, the
result of this transformation is the same as if the rotations corresponding to the factors
of $\mathbf{q}$ had been applied starting with the rightmost factor, and continuing from
right to left. To verify that this is the case, the identity
$(q_1 q_2)^{-1} = q_2^{-1} q_1^{-1}$ is useful.

\subsection{Quaternion integration}
In a dynamics simulation, the angular momentum of a body is determined by solving the
differential equation of motion, given the torques acting on the body. Similarly, the
orientation of the body should be found by numerical integration, given the angular
velocity. Angular velocity or momentum can be represented easily using a 3D vector whose
direction is the axis of rotation and whose magnitude is the scalar quantity (e.g. in
radians per second). As we have seen, such a concise representation is not possible for
orientation, which leads to some complications.

At this point it might be helpful to consider a different view on quaternions. The set of
unit quaternions, defined by equation~\ref{unitQuat}, can be understood as the surface of
a unit sphere in a four-dimensional vector space. Each point on this hypersphere corresponds
to a particular quaternion. It also turns out that each pair of opposite points on this
sphere represent exactly the same rotation; thus all possible rotations are contained in
one hemisphere, no matter where the sphere is cut in half.

The instantaneous rate of change of a quaternion $\mathbf{q}$ over time is usually given
in the literature as
\begin{equation}
\label{quatRateOfChange}
\dot{\mathbf{q}}(t) = \frac{1}{2}\tilde{\mathbf{\omega}}(t)\mathbf{q}(t)
\end{equation}
where $\tilde{\mathbf{\omega}}$ is the quaternion corresponding to the angular velocity
vector $\mathbf{\omega}$. The space of all possible values of $\dot{\mathbf{q}}$ is
conceptually a three-dimensional subspace tangential to the sphere at the point $\mathbf{q}$
in 4D space (proof in appendix~\ref{quatRateOfChangeOrthogonal}). Therefore adding an
infinitessimal amount of $\dot{\mathbf{q}}$ to $\mathbf{q}$ will preserve the property of
unit magnitude; this is exactly what we should expect of the derivative.

Unfortunately this formula fails in practice since simulations use finite time steps. This
shall be demonstrated using Euler's method; it should, however, be pointed out that more
sophisticated methods like RK4 are equally affected. Consider the value of $\mathbf{q}$ at
the next time step, $\mathbf{q}(t + \Delta t) = \mathbf{q}(t) + \Delta t \dot{\mathbf{q}}(t)$.
For any non-zero $\Delta t$ this point will always lie outside the unit sphere
($||\mathbf{q}(t + \Delta t)|| > 1$). This is usually compensated by re-normalizing the
quaternion after the ODE solving step. Geometrically, this re-normalization can be understood as
drawing a line through the origin and the point $\mathbf{q}(t) + \Delta t \dot{\mathbf{q}}(t)$,
intersecting this line with the unit sphere and replacing $\mathbf{q}(t + \Delta t)$ by this
point of intersection.

Following the tangent to the sphere is a reasonable approximation to following its curve if the
magnitude of $\Delta t \dot{\mathbf{q}}(t)$ is small compared to the curvature of the sphere.
For large time steps or large magnitudes of $\mathbf{\omega}$, however, this gets increasingly
erroneous. Consider the limiting case, a body rotating infinitely fast
($||\omega|| \rightarrow \infty$): after re-normalisation, $\mathbf{q}$ will have moved merely
a quarter of the way around the unit sphere, while it would have been correct to perform an
infinite number of revolutions around the quaternion sphere.

We can solve this problem by introducing an additional factor into the definition of
$\dot{\mathbf{q}}$, replacing equation~\ref{quatRateOfChange}:
\begin{equation}
\label{quatRateOfChangeCorrected}
\dot{\mathbf{q}}_{\Delta t}(t) = \frac{1}{||\mathbf{\omega}||\Delta t}
    \tan\left(\frac{||\mathbf{\omega}||\Delta t}{2}\right)
    \tilde{\mathbf{\omega}}(t)\mathbf{q}(t)
\end{equation}
This formula is derived in appendix~\ref{quatIntegrationDerivation}. Given the simulation time
step as an additional parameter, it ensures that the tangent is always scaled appropriately so
that the correct result is obtained after renormalization. This assumes a constant angular
velocity, which is appropriate since the solver for the equations of linear motion makes an
analoguous assumption. When implementing this formula, care must be taken around the
discontinuities of the $\tan$ function, where numerical instability may occur. These
discontinuities are reached whenever a body performs an odd multiple of half
revolutions during a single time step.
