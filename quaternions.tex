\section{Quaternions\label{quaternions}}
\subsection{The need for Quaternions}
Besides its position, each rigid body in 3D space may have an orientation, in which there are
three degrees of freedom (three independent axes to rotate about).
While the position of a body can be neatly represented using Cartesian coordinates, there is
no obvious best way of describing an orientation. The most common schemes describe
it in terms of a rotation operation which transforms a vector in the body's local
coordinates into world coordinates (or \textsl{vice versa}). But again, there are various different
approaches to representing this rotation, all of which have advantages and disadvantages.

\emph{Euler angles} are probably the most intuitive representation of a 3D rotation, describing
it as a series of three rotations about different axes. These axes are fixed by convention, so it
suffices to specify the three angles of rotation. However, this scheme has a number of
drawbacks~\cite{Saunders:PhD,Shoemake:85}: amongst other things, it is possible that rotation
about one of the axes freezes during an animation (``Gimbal lock'').

\emph{Rotation matrices} are commonly used because they are well understood and
allow efficient combination with other linear transformations (scaling
and shearing~-- also translation if homogeneous coordinates are employed). However, ODEs over
rotation matrices are difficult to implement correctly, since this representation uses nine
numbers (a $3\times3$ matrix) to represent three degrees of freedom, thus introducing six
additional side conditions which must be maintained. Not doing so causes skew through numerical
drift~\cite{Saunders:PhD}.

\emph{Quaternions}~\cite{Shoemake:85,Eberly:01,MathWorld:Quaternion} are a popular alternative
to the two previous sche\-mes, and they are used extensively in this project.

\subsection{Definition and properties}
Mathematically, quaternions can be regarded as numbers with one real part and three
distinct imaginary parts:
\begin{equation}
\q{q} = q_w + q_x\qi + q_y\qj + q_z\qk
\end{equation}
where $q_w$, $q_x$, $q_y$ and $q_z$ are real numbers and \qi{}, \qj{} and \qk{} satisfy
\begin{equation}
\qi^2 = \qj^2 = \qk^2 = \qi\qj\qk = -1.
\end{equation}
From this follows that
$\qi\qj = -\qj\qi = \qk$ and
$\qj\qk = -\qk\qj = \qi$ and
$\qk\qi = -\qi\qk = \qj$.
Note that multiplication is not commutative.

We will also need the conjugate and the inverse of a quaternion. In analogy to
complex numbers, these are given respectively by
\begin{eqnarray}
\overline{\q{q}} & = & q_w - q_x\qi - q_y\qj - q_z\qk \label{quatConjugate}\\
\q{q}^{-1} & = & \frac{\overline{\q{q}}}{\norm{\q{q}}^2} \label{quatInverse}\\
\mathrm{where}\quad\quad \norm{q_w + q_x\qi + q_y\qj + q_z\qk} &=&
    \sqrt{q_w^2 + q_x^2 + q_y^2 + q_z^2} \label{quatMagnitude}
\end{eqnarray}

Sometimes we will need to relate a 3D vector to a quaternion with zero real part.
For a given vector $\ve{u} = (u_1, u_2, u_3)^T$ we define the corresponding
quaternion to be
\begin{equation}
\label{vectorToQuat}
\tilde{\ve{u}} = u_1\qi + u_2\qj + u_3\qk.
\end{equation}

The complex constants \qi{}, \qj{} and \qk{} are required for the algebra only, therefore we can
represent a quaternion as four numbers $(q_w,q_x,q_y,q_z)$. It turns out that a quaternion
with unit magnitude ($\norm{\q{q}} = 1$) neatly represents an arbitrary rotation in
3D space, similarly to the way that an ordinary complex number represents a rotation in the 2D
Argand diagram. The condition $\norm{\q{q}} = 1$ reduces the number of degrees of freedom to
three, as required.

Every unit quaternion represents a rotation of angle $\theta$ about an arbitrary axis.
If the axis passes through the origin and has a direction given by the vector
\ve{a} with $\norm{\ve{a}} = 1$, the quaternion describing this rotation is
\begin{equation}
\label{quatRotation}
\q{q} = \cos\left(\frac{\theta}{2}\right) + \tilde{\ve{a}} \sin\left(\frac{\theta}{2}\right).
\end{equation}
It can easily be verified that this quaternion always has unit magnitude. It shall be assumed
throughout this project that the rotation thus described is clockwise (as seen when looking in
the direction of the vector $\ve{a}$) in a right-hand coordinate system, i.e.\ that it is
given by the ``right-hand rule''.

Two rotations can be concatenated by multiplying their quaternions together. The order in which
these rotations are applied is significant, and quaternion multiplication is not commutative,
so the semantics match. The operations are, however, associative. The quaternion product is
obtained simply by multiplying out the components, observing the rules for multiplying \qi{},
\qj{} and \qk{}:
\begin{eqnarray}
\lefteqn{(p_w + p_x\qi + p_y\qj + p_z\qk)(q_w + q_x\qi + q_y\qj + q_z\qk) = } \nonumber\\*
&& (p_w q_w - p_x q_x - p_y q_y - p_z q_z) + 
   (p_w q_x + p_x q_w + p_y q_z - p_z q_y) \,\qi + \nonumber\\*
&& (p_w q_y + p_y q_w + p_z q_x - p_x q_z) \,\qj + 
   (p_w q_z + p_z q_w + p_x q_y - p_y q_x) \,\qk \label{quatProduct}
\end{eqnarray}

To rotate a vector $\ve{v} = (v_1, v_2, v_3)^T$ by a quaternion \q{q}, we first
convert it into its corresponding quaternion $\tilde{\ve{v}}$ as defined in
equation~\ref{vectorToQuat} and then calculate the quaternion product
\begin{equation}
\label{quatTransform}
\tilde{\ve{v}}' = \q{q}\tilde{\ve{v}}\q{q}^{-1}
\end{equation}

If we expand this formula, we find that the real part of the result is always zero, and
that the rotated vector $\ve{v}' = (v_1', v_2', v_3')^T$ corresponds
to $\tilde{\ve{v}}'$ (i.e.\ $\ve{v}'$ is contained in the three complex parts of
the quaternion product).

Some authors, notably Shoemake~\cite{Shoemake:85}, choose to define the product in
equation~\ref{quatTransform} in
the reverse order. The choice is a matter of convention, since it merely changes the effect
of this operation from being a clockwise to a counter-clockwise rotation. I chose the clockwise
convention because it is consistent with the usual definition of the angular velocity vector
in physics.

Observe that under this convention, if \q{q} is itself a product of quaternions
$\q{q} = \q{q}_n \q{q}_{n-1} \cdots \q{q}_1$, the result is that of first applying the
$\q{q}_1$ rotation, then $\q{q}_2$ etc. In other words, the rotations in a quaternion product
are applied from right to left. To verify that this is the case, the identity
$\overline{\q{p}\, \q{q}} = \q{\overline{q}}\; \q{\overline{p}}$
is useful~\cite{MathWorld:Quaternion}.

\subsection{Quaternion integration}

We have seen that given the torques on a body, a numerical ODE solver can treat each component
of the vector separately to obtain the new angular momentum. Now that we are representing
orientation as a quaternion, how do we compute the change in orientation given the body's angular
velocity?

The instantaneous rate of change of a quaternion \q{q} over time is
usually~\cite{BaraffWitkin:97,Eberly:04,Saunders:PhD} given as
\begin{equation}
\label{quatRateOfChange}
\dot{\q{q}}(t) = \frac{1}{2}\tilde{\ve{\omega}}(t)\q{q}(t)
\end{equation}
where $\tilde{\ve{\omega}}$ is the quaternion corresponding to the angular velocity
vector $\ve{\omega}$. The quaternion $\dot{\q{q}}$ can be fed into an ODE solver which can handle
its four components separately. However, when this is done it is observed that the new orientation
$\q{q}'$ no longer has unit magnitude. This is an inherent property of the definition in
equation~\ref{quatRateOfChange}, and not, as sometimes claimed~\cite{Eberly:04}, merely a matter
of numerical round-off (proof in appendix~\ref{quatIntegrationMagnitude}). Usually this problem
is `solved' by renormalizing the quaternion:
\begin{equation}
\q{q}(t + h) = \frac{\q{q}'}{\norm{\q{q}'}} \quad\quad\mathrm{where}\quad
    \q{q}' = \q{q}(t) + h\dot{\q{q}}(t)
\end{equation}
(using Euler's method for clarity; $\q{q}'$ would be appropriately redefined when using e.g.\ RK4).
This `solution' seemed quite \textsl{ad hoc} to me, and I demonstrated that it returns
erroneous results if the angular velocity is large (see appendix~\ref{quatNormalization}).
I derive an exact algorithm for quaternion integration in appendix~\ref{quatProofs}. Such
an algorithm may be important in aerospace applications, as the NASA patent~\cite{NASA:00}
suggests.
