The preparatory work for this project splits into two parts:
\begin{itemize}
\item Setting up and learning to use the software environment for development. This is described
    in section~\ref{softwarePreparation}.
\item Surveying literature, learning the theoretical background and understanding the existing
    algorithms for my task. The outcome of this preparation shall be discussed first.
\end{itemize}

\section{Simulation outline}

\subsection{Rigid body dynamics}

Purpose of a dynamics simulation is to calculate the behaviour of a physical system over time.
The equations of motion for simple systems can often be solved analytically, but they very
rapidly become intractable as systems become more complex. Simulation is often the best
method to numerically find an approximate solution to these equations.

This project is concerned with the simulation of rigid bodies. These are good abstractions of
most solid objects we come across in everyday life: they cannot change their shape, have no
forces of their own, collide instantaneously, and cannot break. In this project we consider
rigid bodies in three-dimensional space.

At a particular point in time, a rigid body is fully determined by four quantities: the
\emph{position} of its centre of mass, its \emph{orientation}, and its \emph{linear} and
\emph{angular velocities}. These values usually change over time. The representation of
orientation is discussed in section~\ref{quaternions}; each of the others can be
represented as a three-dimensional vector. By convention we will use a right-handed Cartesian
coordinate system.

To characterise the body's dynamic behaviour, we also need to know its \emph{inertial mass},
a scalar quantity, and its \emph{moment of inertia}, which is a rank-2 tensor (commonly written
as a $3\times3$ matrix). While the mass stays constant, the moment of inertia may
depend on the body's orientation~-- it is, however, constant when expressed with respect to the
body's \emph{principal axes}~\cite{Feynman:63}.

Angular velocity appears to be a straightforward way of describing the body's rotation: the
direction of the vector gives the axis of rotation, while its magnitude is the rate of rotation
(most commonly given in radians per second). Unfortunately, in an asymmetric body, the angular
velocity may vary over time even if there are no external influences on the body. It is therefore
sometimes more convenient to use \emph{angular momentum} instead, which is conserved in the
absence of torques.

\begin{table}[hbt]
\begin{tabular}{|l|l|l|} \hline
& \emph{Linear} & \emph{Angular} \\\hline
Resistance to change & Mass $m$ & Moment of inertia \m{I} \\\hline
Stationary state & Centre of mass & \emph{see section \ref{quaternions}} \\
                 & position \ve{r} & \\\hline
Velocity & Linear velocity $\ve{v} = \dot{\ve{r}}$ & Angular velocity \ve{\omega} \\\hline
Momentum & Linar momentum     & Angular momentum           \\
         & $\ve{p} = m\ve{v}$ & $\ve{L} = \m{I}\ve{\omega}$\\\hline
External influence & Force $\ve{F} = \dot{\ve{p}}$ & Torque $\ve{\tau} = \dot{\ve{L}}$ \\\hline
\end{tabular}
\caption{Summary of the physical quantities describing a rigid body\label{rigidBodySummary}}
\end{table}

Forces and torques acting on a body may change the momenta of a body as described in
table~\ref{rigidBodySummary}. A force \ve{F} may be applied to any point $\ve{r}'$ of
a body. We can treat this as if the force had been applied to the centre of mass \ve{r}, and add
an additional torque given by $\ve{\tau} = (\ve{r}' - \ve{r})\times\ve{F}$. If multiple
forces and torques are applied, all forces (applied to the centre of mass) may be added into
a single vector, and similarly all torques may be added.

So, in a nutshell, a rigid body simulation is a solution of the differential equations of
motion. Forces are integrated over time to find linear momentum, torques are integrated to find
angular momentum. From each of these we calculate linear and angular velocities, which are in turn
integrated to find the position and orientation over the course of time. It is important that we
integrate over torques (and not angular accelerations), otherwise the simulation will not
correctly conserve angular momentum.

\subsection{Solving ordinary differential equations}

-- to be written --

Only simple cases can be solved analytically. For example, if a constant force is applied to a
body, its displacement is a quadratic (parabolic) function of time; if the force is inversely
proportional to the displacement, we get simple harmonic oscillation.
